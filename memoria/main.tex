\documentclass{article}
% chktex-file 1

\usepackage{graphicx} 
\usepackage[utf8]{inputenc}
\usepackage[a4paper, margin=1in]{geometry}
\usepackage[spanish, es-tabla]{babel}
\usepackage{fancyhdr}
\usepackage[hidelinks]{hyperref}
\usepackage{pdfpages}
\usepackage[style=ieee]{biblatex}
\usepackage{csquotes}
\usepackage{booktabs}
\usepackage[parfill]{parskip}
\usepackage{longtable}
\usepackage{float}
\usepackage{amstext}
\usepackage{mathtools}
\usepackage[titletoc, title]{appendix}
\usepackage{multirow}
\usepackage{listings}
\usepackage{csquotes}
\usepackage{caption}
\usepackage{subcaption}
% \usepackage{tikz}
\usepackage{epigraph}
\usepackage{bytefield}
\usepackage{titlesec}

\setcounter{secnumdepth}{4}

\titleformat{\paragraph}
{\normalfont\normalsize\bfseries}{\theparagraph}{1em}{}
\titlespacing*{\paragraph}
{0pt}{3.25ex plus 1ex minus .2ex}{1.5ex plus .2ex}

\addto\captionsspanish{%
  \renewcommand\appendixname{Anexo}
  \renewcommand\appendixpagename{Anexos}
  \renewcommand\appendixautorefname{Anexo}
}
% \renewcommand{\arraystretch}{1.2} 
\newcommand{\isc}{$I^2C$}
\addto\extrasspanish{%
  \def\subsectionautorefname{Apartado}%
  \def\subsubsectionautorefname{Subapartado}%
}
\DeclarePairedDelimiter{\ceil}{\lceil}{\rceil}
\DeclarePairedDelimiter{\floot}{\lfloor}{\rfloor}

% Bibliografia
\addbibresource{ISE.bib}

\setcounter{biburllcpenalty}{7000}
\setcounter{biburlucpenalty}{8000}

\makeatletter
\def\thebibliography#1{\section{\refname\@mkboth
{\uppercase{\refname}}{\uppercase{\refname}}}\list
{\@biblabel{\arabic{enumiv}}}{\settowidth\labelwidth{\@biblabel{#1}}%
\leftmargin\labelwidth
\advance\leftmargin\labelsep
\usecounter{enumiv}%
\let\***@enumiv\@empty
\def\theenumiv{\arabic{enumiv}}}%
\def\newblock{\hskip .11em plus.33em minus.07em}%
\sloppy\clubpenalty4000\widowpenalty4000
\sfcode`\.=1000\relax}
\makeatother

% Numeración de páginas
\makeatletter
\newcommand\frontmatter{%
    \cleardoublepage
  %\@mainmatterfalse
  \pagenumbering{roman}}
\newcommand\mainmatter{%
    \cleardoublepage
 % \@mainmattertrue
  \pagenumbering{arabic}}
\newcommand\backmatter{%
  \if@openright
    \cleardoublepage
  \else
    \clearpage
  \fi
 % \@mainmatterfalse
   }
\makeatother

\renewcommand\lstlistingname{Fragmento}
\renewcommand\lstlistlistingname{Índice de fragmentos}

\definecolor{dkgreen}{rgb}{0,0.6,0}
\definecolor{gray}{rgb}{0.5,0.5,0.5}
\definecolor{mauve}{rgb}{0.58,0,0.82}

\lstset{frame=tb,
  language=c,
  aboveskip=3mm,
  belowskip=3mm,
  showstringspaces=false,
  columns=flexible,
  basicstyle={\small\ttfamily},
  numbers=none,
  numberstyle=\tiny\color{gray},
  keywordstyle=\color{blue},
  commentstyle=\color{dkgreen},
  stringstyle=\color{mauve},
  breaklines=true,
  breakatwhitespace=true,
  tabsize=3
}



\title{Procesador de audio en tiempo real}
\author{Rubén Agustín \\ David Andrino \\ Estela Mora \\ Fernando Sanz}
\date{Primavera 2024}

\begin{document}
\frontmatter

\begin{titlepage}
    \raggedleft
    \rule{1pt}{\textheight}
    \hspace{0.05\textwidth}
    \parbox[b]{0.9\textwidth}{
            {\Huge\bfseries Real-Time Audio \\[5px] Processor}\\[\baselineskip] % Title
            {\Large\textit{Memoria del proyecto}}\\[7\baselineskip] % Subtitle or further description
        \vspace{0.45\textheight}
        
        {\Large\textsc{Rubén\ Agustín}}\\[0.5\baselineskip]
        {\Large\textsc{David Andrino}}\\[0.5\baselineskip]
        {\Large\textsc{Estela Mora}}\\[0.5\baselineskip]
        {\Large\textsc{Fernando Sanz}}\\
        \vspace{0.05\textheight}
        
        {\noindent\large Ingeniería de Sistemas Electrónicos}\\
        {\noindent\large Primavera 2024}\\
    }

\end{titlepage}

\tableofcontents
\listoffigures
% \listoftables
% \lstlistoflistings
\mainmatter

\pagestyle{fancy}
\lhead[]{\url{https://github.com/David-Andrino/ise-rtap}{\textbf{Real-Time Audio Processor}}}
\rhead[]{\textbf{Memoria del proyecto}}
\cfoot{\thepage}

\section{Introducción}

\subsection{Especificaciones iniciales del sistema a diseñar y construir}

El proyecto consiste en el desarrollo de un sistema de procesamiento de audio en tiempo real con un sistema centrado en un microprocesador.

El sistema permitirá entrada de audio a través de un módulo de radio FM y la lectura de canciones de una tarjeta microSD externa. Tras capturar dicho audio, se le aplicará un procesamiento digital implementado mediante la librería CMSIS DSP y se reproducirá por un altavoz o unos auriculares.

La interacción con el sistema se realizará a través de una pantalla táctil o una interfaz web, la cual requiere conexión a través de Ethernet. Dicha interfaz permite seleccionar la entrada de audio, controlar el volumen y ecualización de audio y elegir la salida. Además, permite controlar la emisora de radio sintonizada.

El sistema almacenará los parámetros seleccionados (entrada, salida y filtros) y una lista de emisoras favoritas en la tarjeta microSD, los cuales se cargarán al iniciar el equipo.

El sistema será completamente autónomo, contando con una batería con su correspondiente circuitería de carga, protección y medición de consumo. Se ofrecerá información sobre la batería en la interfaz gráfica del sistema. Además, el sistema contará con un modo de bajo consumo para alargar la duración de dicha batería.

Para la selección de canciones se permitirá el uso de tarjetas NFC preconfiguradas con canciones o emisoras preconfiguradas, para la interacción sin interfaz gráfica. Por último, el sistema utilizará el RTC integrado en la placa para mantener la hora y el protocolo SNTP para la sincronización.

\subsection{Especificaciones finales del sistema diseñado y construido}

El sistema permite la entrada de audio a través del Sintonizador FM y la lectura de canciones de una tarjeta microSD externa  mediante un reproductor MP3. Tras capturar dicho audio, se le aplica el procesamiento digital implementado mediante la librería CMSIS DSP y se reproduce por un altavoz o unos auriculares.

La interacción con el sistema se realiza a través de la pantalla táctil o mediante la interfaz web, la cual requiere conexión a través de Ethernet. Dicha interfaz nos permite seleccionar la entrada de audio, controlar el volumen y aplicar la ecualización deseada al audio y elegir la salida. Además, permite seleccionar y controlar la emisora de radio sintonizada.

El sistema permite la lectura de los parámetros de configuración seleccionados (entrada, salida, filtros y volumen) y una lista de canciones de la tarjeta uSD, escritos mediante un ordenador.

Por otra parte, el sistema permite almacenar un mapa de frecuencia a emisora en el almacenamiento de la propia tarjeta.

El sistema no es completamente autónomo,ya que, aunque cuente con una batería con su correspondiente circuitería de carga, protección y medición de consumo, el sistema necesita una conexión constante mediante el \texttt{ST\_Link}. Se ofrece información sobre el consumo medido en la interfaz gráfica del sistema. Además, el sistema cuenta con un modo de bajo consumo para alargar la duración de dicha batería.

Para la selección de canciones y emisoras, se permitirá el uso de tarjetas NFC preconfiguradas con canciones o emisoras preconfiguradas, para la interacción sin interfaz gráfica. Por último, el sistema utiliza el RTC integrado en la propia placa para mantener la hora y fecha actual mediante la sincronización son un servidor SNTP.

A continuación se adjunta una imagen del sistema completo, con todos su componentes y en su total funcionamiento:

\begin{figure}[!hp]
    \centering
    \includegraphics[width=\textwidth]{images/1/Foto_Sistema.jpg}
    \caption{Sistema Completo}
    \label{fig:1-Sistema_Completo}
\end{figure}
\section{Desarrollo de subsistemas}

Para este proyecto hemos desarrollado dos subsistemas analógicos propios, sobre dos PCB distintas, una de alimentación y un amplificador de audio. Además, hemos utilizado tres subsistemas ya existentes para la recepción de audio, la reproducción de música en MP3 y la lectura de información NFC de un dispositivo móvil.

\subsection{La placa}
Para este proyecto se ha utilizado la placa \texttt{STM32F769NI-DISCO}, que es una placa de desarrollo comercializada por STM. Dispone de un procesador ARM Cortex M7, que puede funcionar a una frecuencia de hasta 216 MHz. 

Dispone de pines dispuestos de una forma adecuada para ser compatible con Arduino Uno.

Entre los periféricos presentes en la placa, los más importantes que utilizamos son:
\begin{itemize}
    \item \textbf{La pantalla:} Dispone de una pantalla LCD táctil de 4.3 pulgadas, accesible a través del periférico DSI, con una resolución de 800 x 480 píxeles, con una profundidad del  color de 16 bits, lo que resulta en un total de $2^{16}$ colores posibles de representar. En concreto, se asignan 5 bits para las componentes roja y azul, siendo los 6 restantes para la verde, que es la más importante.
    \item \textbf{La DMA2:} La pantalla restringe el uso de las DMA, siendo la 2 la única opción posible, puesto que es la única que tiene la opción de \textit{Memory-To-Memory}. Cualquier canal y flujo (\textit{Channel} y \textit{Stream}) son de libre elección.
    \item \textbf{Los ADCs:} La placa dispone tres ADCs, que funcionan a una frecuencia de 48 kHz, de los cuales nuestro proyecto utiliza dos: el ADC1 se utiliza para medir el consumo, y el ADC3 se utiliza como pin de entrada para el audio. 
    \item \textbf{El DAC:} La placa dispone de un DAC de 2 canales. En la documentación de la placa no se anuncia esta capacidad, únicamente se puede encontrar el la documentación del microcontrolador. El DAC también funciona a 48kHz, y se sincroniza con el ADC, como se explica en \autoref{para:temporizacion}.
    \item \textbf{El I2C:} Utilizamos un I2C para la comunicación con el módulo del NFC.
    \item \textbf{La SD:} Dispone de un puerto para introducir una tarjeta SD, a través del periférico SDMMC.
\end{itemize} 

\subsubsection{Algunas consideraciones}
En algunos proyectos, hemos intentado utilizar la sentencia \texttt{printf}, pero a priori parecía una tarea imposible. Revisando la documentación, constatamos que el pin \texttt{SW0} está desconectado en la configuración por defecto. Para que funcione dicha función, hay que soldar \texttt{R92}, que conecta \texttt{SW0} con el pin adecuado. Esta resistencia funciona como un jumper, ya que el valor anunciado por el fabricante es de 0 $\Omega$. Tras sujetar un cable de forma manual, conseguimos visualizar la salida de la \texttt{UART} desde el panel del debugger.
\subsection{Subsistema de alimentación}

El subsistema de alimentación es el encargado de permitir que el sistema se alimente a través de baterías. También permite cargar la batería e incluso ambas acciones simultáneamente.

Como batería hemos decidido utilizar una batería de ion de litio 18650, un modelo bastante estándar en las aplicaciones integradas.  Concretamente, hemos decidido utilizar el modelo \texttt{INR18650-29E} de Samsung, que tiene una tensión nominal de $3.7\ V$ y una capacidad de $2900\ mAh$.

Sin embargo, este tipo de baterías tienen unos requisitos de carga muy estrictos. Contando con que la batería parte de un estado correcto (no por debajo del límite de tensión segura), se debe cargar primero con un método de corriente constante hasta alcanzar una tensión cercana a la máxima y después cambiar a un método de tensión constante hasta finalizar la carga \cite{BU409ChargingLithiumion}. Un ejemplo de ciclo de carga se puede ver en la \autoref{fig:2-1-cicloCarga}

\begin{figure}[h]
    \centering
    \includegraphics[width=0.5\textwidth]{images/2/2-1/cargaBateria.png}
    \caption{Ciclo de carga de una batería de ión de litio}
    \label{fig:2-1-cicloCarga}
\end{figure}

Si no se respeta el ciclo de carga de estas baterías, se tiene una alta probabilidad de que falle de forma violenta, siendo incluso un peligro de incendio. 

\subsubsection{Circuito de carga}

Para facilitarnos la tarea de respetar este ciclo hemos utilizado una solución integrada de Texas Instruments, el \texttt{BQ25606}, un cargador de una celda de litio que soporta hasta 3 amperios de corriente. \cite{BQ25606DataSheet}

Gracias a este circuito integrado y bastantes componentes externos, podemos diseñar un circuito que, a partir de una tensión de entrada entre 5 y 12 voltios y utilizando un convertidor reductor, carga la batería adecuadamente y de forma segura. 

La entrada al circuito puede ser a través de un conector \textit{Jack} de alimentación o un conector \textit{Micro USB}. Además, si se utiliza el segundo conector, el integrado se encarga de negociar el protocolo de carga rápida con la fuente de alimentación para incrementar la corriente de entrada y mejorar la potencia de carga. 

Además, si el circuito está conectado a la alimentación y la fuente tiene suficiente capacidad, la corriente de salida se obtiene de la entrada en lugar de la batería, gracias a la tecnología PowerPath de TI. Esta tecnología además permite balancear las tres corrientes, por lo que si el sistema requiriera de más corriente de la que la fuente de alimentación pudiera proveer, se obtendría también de la batería realizando un esfuerzo coordinado entre la fuente y la batería. Si por el contrario la fuente puede ofrecer más corriente de la que se está solicitando en la salida, se utiliza este excedente para cargar la batería.

Este circuito cuenta además con dos indicadores LED que informan sobre si la fuente de alimentación está conectada y las tensiones son correctas (verde en nuestro circuito) y si se está cargando la batería o hay algún fallo (roja en nuestro circuito). La segunda luz se mantiene encendida mientras se está cargando, se apaga cuando se ha finalizado la carga y parpadea a $1 Hz$ de frecuencia si hay algún error.

Para ofrecer todas estas características, el circuito integrado consta de una máquina de estados finitos y múltiples comparadores de error. Con esta inteligencia conmuta tres transistores para conectar la batería y para gestionar el convertidor conmutado síncrono que se construye. Se puede ver un diagrama de bloques resumido del circuito (obtenido de la hoja de catálogo) en la \autoref{fig:2-1-bloquesInternosBQ25606}

\begin{figure}[h]
    \centering
    \includegraphics[width=0.5\textwidth]{images/2/2-1/BQ25606Bloques.png}
    \caption{Diagrama de bloques interno del BQ25606}
    \label{fig:2-1-bloquesInternosBQ25606}
\end{figure}

Este circuito ofrece a su salida una tensión aproximadamente igual a la de la batería durante el funcionamiento normal, por lo que está alrededor de los $3.7 V$. Sin embargo, si la tensión de la batería cae por debajo de $3.5 V$, el circuito se encarga de ofrecer dicha tensión a la salida, por lo que nunca bajará de dicho valor (mientras la batería puede ofrecer corriente y no esté descargada).

El circuito integrado ofrece también la posibilidad de utilizar una batería con sensor de temperatura integrado, pero no vamos a utilizarlo debido al incremento en coste de la misma.

En la hoja de catálogo del integrado se ofrece información sobre el proceso de diseño de un circuito alrededor de dicho integrado. Además, se tiene una nota de aplicación sobre el diseño de circuitos alrededor de este integrado \cite{texasinstrumentsDesigningStandaloneSingle}:

\begin{itemize}
    \item Inductancia de $2.2 \mu F$ para reducir el rizado de corriente.
    \item Pin \texttt{VSET} flotante para tensión de carga máxima de $4.208 V$.
    \item Divisor de tensión de dos resistencias de $10 k\Omega$ en \texttt{TS} para no utilizar divisor de tensión.
    \item Resistencia de $165 \Omega$ en \texttt{ILIM} para limitar la corriente de entrada cuando no se negocia carga rápida a $3 A$ (normalmente se utiliza mucho menos).
    \item Resistencia de $487 \Omega$ en \texttt{ICHG} para limitar la corriente de carga máxima a $1.4 A$ como se indica en las especificaciones de la batería.
    \item El resto de componentes se toman de la Figura 10-1 de la hoja de catálogo
\end{itemize}

Por tanto, esta parte del circuito queda como se puede ver en la \autoref{fig:circuito-carga-final}.

\begin{figure}[h]
    \centering
    \includegraphics[width=\textwidth]{images/2/2-1/circuitoCarga.png}
    \caption{Subcircuito de carga de batería}
    \label{fig:circuito-carga-final}
\end{figure}

\subsubsection{Convertidor elevador}\label{subsubsec:convertidor_elevador}

La placa que utilizamos especifica una tensión de entrada de $7 V$ a $12 V$ si se utiliza el pin $V_{in}$ inferior. Preferimos utilizar este método ya que la entrada de $5 V$ no tiene protección y se puede dañar la placa si no se realiza todo el proceso adecuadamente.

Experimentalmente hemos notado que la placa utiliza la misma corriente de entrada para cualquier valor de tensión, por lo que seguramente utilice un convertidor de tensión lineal internamente para generar las tensiones. Por tanto, hemos preferido tomar el valor más pequeño de las tensiones de entrada para reducir la pérdida de potencia.

La tensión de salida del circuito de carga oscila entre los $3.5 V$ y los $4.2 V$, por lo que hemos diseñado un convertidor conmutado elevador o \textit{Boost Converter} para obtener dicha tensión a partir de la salida del cargador.

Un circuito convertidor elevador está compuesto básicamente por una bobina, un transistor y un controlador PWM.\ Como controlador PWM hemos utilizado otra solución integrada de Texas Instruments debido a la alta precisión que nos permite tener, ya que un fallo en este circuito podría dañar seriamente al resto del sistema. 

\subsubsection{Medidor de consumo}
\subsubsection{Diseño de PCB}
\clearpage
\subsection{Subsistema de audio}

El subsistema analógico de audio tiene tres funciones principales:
\begin{enumerate}
    \item Multiplexar las dos entradas de audio (MP3 y Radio) a un ADC del microcontrolador
    \item Multiplexar la salida proveniente del DAC a una de las dos salidas de audio (Altavoces o auriculares)
    \item Amplificar la salida de audio seleccionada
\end{enumerate}

\subsubsection{Divisor de rail}
Este sistema se alimenta directamente desde el módulo de alimentación explicado en el apartado anterior. Por ello, cuenta con una entrada unipolar de 7 voltios. Sin embargo, los circuitos de audio necesitan alimentación bipolar para funcionar. Por ello, hemos decidido implementar un circuito divisor de rail, que genera una tensión intermedia que se utilizará como nueva referencia del circuito, obteniendo una alimentación virtual de $\pm$3.5 V.

El circuito utilizado es un divisor de tensión de precisión implementado mediante el amplificador operacional \texttt{OP07}. Este amplificador operacional tiene muy bajo error estático y además cuenta con dos terminales para el ajuste del offset mediante un potenciómetro, por lo que es ideal para nuestra aplicación.

Sin embargo, los amplificadores operacionales permiten muy poca corriente a través de su terminal de salida, por lo que se necesita un circuito que permita incrementar la corriente de salida o entrada de dicho circuito sin afectar demasiado a la salida. Para ello, se utiliza una topología \textit{Push-Pull} en la que se utilizan dos pares de Darlington, es decir, cuatro transistores, para incrementar la capacidad de corriente del circuito. 

Al introducirlos dentro del lazo de realimentación, se elimina el efecto de las tensiones de base y se elimina el ruido que pudieran introducir, pero a cambio introducen la posibilidad de inestabilizar el circuito. Teniendo esta posibilidad en cuenta, incluimos la posibilidad de soldar un condensador entre el terminal de salida del amplificador y la alimentación negativa del circuito, para introducir un polo que compense la estabilidad. Sin embargo, hemos acabado no necesitando utilizarlo. 

Por tanto, el circuito generador de tierra virtual o divisor de rail queda como se puede ver en la \autoref{fig:2-2-tierra-virtual}.

\begin{figure}[h]
    \centering
    \includegraphics[width=0.5\textwidth]{images/2/2-2/circuitoDivisorRail.png}
    \caption{Circuito divisor de raíl}
    \label{fig:2-2-tierra-virtual}
\end{figure}

\subsubsection{Habilitación del circuito}

Para eliminar el consumo parásito del circuito cuando el sistema entre en el modo bajo consumo, se ha implementado un subcircuito de habilitación el cual permite encender o apagar el resto de subsistema (aunque finalmente el consumo parásito es muy pequeño).

Dicho circuito consiste en un transistor MOSFET de canal N en la alimentación, que permite cortar o dejar pasar la alimentación. Además, la baja impedancia de conducción del transistor permite que no haya casi pérdidas de potencia en el transistor. Sin embargo, ya que para cortar el transistor se necesita polarizar la puerta con una tensión próxima a 7 voltios y soportar las corrientes de los transitorios de conmutación, se utiliza otro transistor con una resistencia de \textit{pull-up} para adaptar los niveles los GPIO y reducir la corriente necesaria. Esto tiene el efecto añadido de invertir la polaridad de la habilitación que junto a la inversión del canal N se anulan, provocando que un nivel alto en el GPIO habilite el circuito. 

Por tanto, el circuito final es el que se ve en la \autoref{fig:2-2-circuito-habilitacion}.

\begin{figure}[h]
    \centering
    \includegraphics[width=0.5\textwidth]{images/2/2-2/circuitoHabilitacion.png}
    \caption{Circuito de habilitación}
    \label{fig:2-2-circuito-habilitacion}
\end{figure}

\subsubsection{Multiplexación de audio}

Para la multiplexación de audio se va a utilizar un multiplexor integrado. Inicialmente tratamos de diseñar un circuito que multiplexara los caminos de audio mediante componentes discretos, encontrando la estructura de la Puerta de Transmisión \cite{TransmissionGate}, como la que se puede ver en la \autoref{fig:2-2-puerta-transmision}

Sin embargo, todas las estructuras discretas que encontramos necesitan una familia de transistores de efecto de campo en los que el canal no está unido internamente al sustrato, permitiendo cargar la capacidad puerta-canal sin afectar a la tensión del camino drenador-surtidor. 

El principal problema de estos transistores es su elevado precio y muy poca variedad, siendo casi imposible encotrarlos. Además, generalmente se utilizan en aplicaciones de alta potencia por lo que su rendimiento para aplicaciones de baja señal suele ser bastante pobre.

\begin{figure}[h]
    \centering
    \includegraphics[width=0.3\textwidth]{images/2/2-2/puertaTransmisión.png}
    \caption{Puerta de transmisión con transistores con canal desconectado}
    \label{fig:2-2-puerta-transmision}
\end{figure}

Finalmente, descartamos la idea de utilizar componentes discretos y utilizamos una solución integrada. Por tanto, utilizamos el multiplexor analógico \texttt{CD4053BC}. \cite{CD4053BDataSheet}

Este multiplexor cuenta con tres canales en configuración \texttt{SPDT}, por lo que cada canal tiene un terminal en un extremo y dos en el otro. Este multiplexor cuenta con la ventaja de ser bidireccional, cosa de la que muchos otros carecen y es fundamental para nuestra funciononalidad.

Este multiplexor se utiliza para conectar las dos entradas de audio, que provienen de conectores Jack de audio de 3.5 mm a un GPIO que se conecta internamente a un ADC de la placa y para conectar otro GPIO que se conecta internamente a un DAC a las entradas de los dos caminos de amplificación de audio. 

\subsubsection{Cambiador de nivel lógico}

Un error que cometimos es no tener en cuenta la tensión de habilitación necesaria para conmutar un canal del multiplexor, por lo que los 3.3V de salida de un GPIO no son suficientes para cambiar el canal del multiplexor. Esto provoca que se esté siempre seleccionado el canal correspondiente al nivel bajo.

Para solucionar esto, hemos construido un circuito cambiador de nivel lógico que adapta los 3.3 V de la placa a los 7 V necesarios para conmutar el multiplexor (realmente el mínimo es aproximadamente 5V).

La solución que hemos pensado consiste en un inversor lógico TTL, que consiste en un transistor bipolar NPN con una resistencia de \textit{Pull-up}. La única desventaja es la inversión de nivel, pero se corrige fácilmente en el software.

Se puede ver el diagrama de nuestra solución en la \autoref{fig:2-2-cambiador-nivel}, en la que se muestra un cambiador. Hemos soldado dos de ellos en una placa de prototipado, que se puede ver en la \autoref{fig:2-2-foto-cambiador}.

\begin{figure}[h]
    \centering
    \includegraphics[width=0.5\textwidth]{images/2/2-2/circuitoCambiadorNivel.png}
    \caption{Circuito cambiador de niveles}
    \label{fig:label}
\end{figure}

\begin{figure}[h]
    \centering
    \includegraphics[width=0.25\textwidth]{images/2/2-2/cambiadorNivel.jpg}
    \caption{Foto del circuito cambiador de nivel}
    \label{fig:2-2-foto-cambiador}
\end{figure}
\subsubsection{Amplificador de audio para los auriculares}

La salida del DAC de la placa es una señal entre 0 y 3.3 V con 12 bits de resolución. Por tanto, el audio que se genere tiene una componente continua que se debe eliminar. 

Para eliminar esta componente continua hemos implementado una configuración de filtro paso alto mediante un filtrado pasivo y un seguidor de tensión realizado con un amplificador operacional. En el camino de realimentación del amplificador operacional se coloca una resistencia para anular la tensión de error de \textit{offset} del amplificador operacional.

También se añade la misma estructura de transistores que en el circuito generador de tierra virtual, que ahora al estar también dentro del lazo de realimentación cuentan con la ventaja de que se anula la distorsión de cruce. 

Los valores elegidos para los componentes del filtro son una resistencia de $100\ k\Omega$ y un condensador de $100\ nF$. Con ello, conseguimos una frecuencia de corte de:

\[
    f_c = \frac{1}{2\pi RC} \approx 16\ Hz    
\]

Elegimos este valor ya la banda de audición humana máxima es de $20$ Hz a $20$ kHz.

Se puede ver un diagrama de la solución que montamos en la \autoref{fig:2-2-amp-cascos}.

\begin{figure}[h]
    \centering
    \includegraphics[width=0.7\textwidth]{images/2/2-2/circuitoAmplificadorCascos.png}
    \caption{Circuito amplificador de auriculares}
    \label{fig:2-2-amp-cascos}
\end{figure}

Sin embargo, al montar y probar el circuito detectamos que tenía un problema de inestabilidad para una frecuencia, lo cual es un problema común en los circuitos realimentados. Esto ocurre ya que la introducción del los transistores añade modificaciones impredecibles a la ganancia de lazo, provocando un muy molesto zumbido en los altavoces.

La solución que encontramos a este problema es la realimentación parcial mediante un condensador entre la salida del amplificador operacional y la base de los transistores. El tamaño del condensador afecta directamente a la reducción del ruido, cuanta más capacidad mejor lo elimina ya que hace una realimentación más directa. Sin embargo, cuanta mayor capacidad se le aplique, más se aprecia el efecto de la distorsión de cruce, la cual era eliminada al hacer la realimentación a la salida.

Experimentalmente probamos valores distintos determinando que el mejor balance entre ruido y distorsión de cruce se tiene para un valor aproximado de $33 \mu F$. Sin embargo, este valor no es propio del circuito sino de las tolerancias y efectos parásitos de los componentes, por lo que podría variar mucho según las circunstancias.

Otra cuestión a la que nos enfrentamos es la cantidad de canales. Inicialmente diseñamos el circuito para ofrecer un solo canal de audio y dirigirlo a ambos canales, pero se pierde bastante calidad por lo que decidimos dejarlo en audio por un solo canal. Si se quisiera obtener audio por ambos canales o incluso estéreo, se debería duplicar este circuito y colocar uno por canal.

\subsubsection{Amplificador de altavoces}

El circuito amplificador de altavoces cuenta con el mismo paso bajo que el amplificador de los auriculares, pero se utiliza otra resistencia para aportar ganancia al circuito. Además, ya que se introduce la rama a tierra se añaden un par de condensadores para realizar un filtrado de alta frecuencia y eliminar el ruido debido a la tensión y corrientes de offset.

La frecuencia de corte del filtro paso bajo es de:
\[
    f_c = \frac{1}{2\pi RC} \approx 21.2\ kHz
\]

Elegida igualmente para eliminar las frecuencias fuera del espectro auditivo humano.

La ganancia se elige para convertir el rango de salida ideal del DAC ($[0, 3.3]\ V$) en el rango máximo ideal de los amplificadores antes de la saturación ($[0, 7]\ V$) aunque la señal de audio no va a llenar el fondo de escala por su reducida amplitud. Por tanto, se elige una ganancia de tensión de:

\[
    A_v = 1 + \frac{R_4}{R_5} = 1.91 V/V
\]

Al igual que los otros dos circuitos, se introducen los transistores en el lazo de realimentación para la corriente, aunque en este caso no es necesario el condensador de realimentación parcial. Se tiene un esquemático de este subcircuito en la \autoref{fig:2-2-amp-altavoz}.

\begin{figure}[h]
    \centering
    \includegraphics[width=0.7\textwidth]{images/2/2-2/circuitoAmplificadorAltavoces.png}
    \caption{Circuito amplificador de altavoces}
    \label{fig:2-2-amp-altavoz}
\end{figure}

\subsubsection{Diseño de PCB}

Todos estos sistemas se han integrado en una única PCB para intentar maximizar la integridad de la señal de audio, lo cual se consigue totalmente si se conecta el circuito sin tener en cuenta el \autoref{subsec:entre-dos-tierras}.

Hemos utilizado conectores Jack hembra de 3.5 mm para las dos entradas de audio y la salida a los auriculares. Un problema de este tipo de conectores es que el número de anillos puede variar en función de si los auriculares tienen o no micrófono y si son mono o estéreo. Si se quiere utilizar unos auricules con micrófono con nuestro sistema, se deben dejar ligeramente extraidos del conector para que haga mejor contacto la banda de tierra y mejorar significativamente el sonido.

La salida de altavoz se realiza a través de un terminal de dos tornillos para facilitar su conexión.

Como ya hemos comentado anteriormente, el circuito de transistores cuenta con un condensador de estabilizacion que finalmente no hemos utilizado. 

Además, hemos añadido la posibilidad de utilizar una red de Zobel, circuito que sirve para linealizar la respuesta en frecuencia de la inductancia intrínseca de los altavoces mediante un capacitor y una resistencia, pero finalmente no la hemos necesitado, por lo que tampoco está soldada.

El circuito cuenta además con un potenciómetro que sirve para ajustar el offset de la tensión de la tierra virtual gracias a los terminales específicos del \texttt{OP07}.

Hemos utilizado componentes SMT para los componentes pasivos y los conectores de audio y THT para los circuitos integrados, transistores y conectores de terminal.

Se puede ver una imagen del circuito finalizado en la \autoref{fig:2-2-circuito-foto} y el esquemático completo en el \autoref{anexo:circuito-audio}.

\begin{figure}[h]
    \centering
    \includegraphics[width=0.5\textwidth]{images/2/2-2/circuito-foto.jpg}
    \caption{Circuito de audio completo}
    \label{fig:2-2-circuito-foto}
\end{figure}
\subsection{Módulo de radio}
El modelo de radio elegido ha sido el Sintonizador FM RDA5807M, mostrado en la \autoref{fig:2-3-Radio},  utilizado anteriormente en la asignatura de Sistemas Basados en Microprocesadores.

Dicho modelo se comunica con el microcontrolador mediante el protocolo \texttt{I2C}. El sintonizador cuenta con un decodificador MPX, salida de audio stereo y un rango de sintonización de 87 MHz a 108 MHz debido a nuestra situación geográfica.

En cuanto a la señal de audio de salida, hemos encontrado que presenta una componente continua de 1.65V y el mismo valor de amplitud, es decir, una señal de 3.3V pico a pico centrada en 1.65V.

Todas las características de dicho sintonizador FM se han obtenido del datasheet ofrecido por el fabricante.

\begin{figure}[h]
    \centering
    \includegraphics[width=0.3\textwidth]{images/2/2-3/Radio.jpg}
    \caption{Sintonizador FM RDA5807M}
    \label{fig:2-3-Radio}
\end{figure}
\subsection{Módulo MP3}
\subsection{Módulo NFC}

Para su desarrollo, hemos utilizado el periférico I2C1, el protocolo RF, la placa \texttt{ANT7-T-M24SR64} \cite{M24SR64YPagWeb} de ST Microelectronics y la aplicación móvil \texttt{NFC Tools} \cite{NFCTools}. Además, a la hora de realizar comprobaciones desarrollando el módulo, hemos utilizado la app \texttt{ST25} \cite{ST25}.

\texttt{ANT7-T-M24SR64}: 
La placa \texttt{ANT7-T-M24SR64} es una placa que incluye un \texttt{M24SR64-Y}. \texttt{M24SR64-Y} es una tag dinámica NFC/RFID, EEPROM de interfaz dual, con protocolos RF e I2C. Se puede operar desde una interfaz I2C, un lector RFID o un teléfono con NFC.

\begin{figure}[h]
    \centering
    \includegraphics[width=0.15\textwidth]{images/2/2-5/M24SR.png}
    \caption{Módulo NFC \texttt{ANT7-T-M24SR64}}
    \label{fig:2-5-modulo-nfc}
\end{figure}

Como se puede observar en la figura anterior, hay 6 pines:

\begin{itemize}
    \item \texttt{VCC}: Alimentación 3.3 V.
    \item \texttt{GND}: Masa.
    \item \texttt{SDA}: Línea de datos del bus I2C.
    \item \texttt{SCL}: Señal de reloj del bus I2C.
    \item \texttt{GPO}: A nivel bajo, RF o I2C está siendo utilizado. A nivel alto, está libre.
    \item \texttt{DIS}: Activación/Desactivación de los comandos RF.
\end{itemize}

\subsection{Entre Dos Tierras}
\label{subsec:entre-dos-tierras}

El mayor problema al que nos hemos enfrentado en este proyecto es la gestión de las tierras en las señales de audio. Inicialmente diseñamos el circuito para que las señales de audio se conectaran entre la entrada de audio y la tierra virtual del circuito analógico, pensando que las señales serían compatibles con la entrada del ADC al estar alimentadas con los mismos niveles de tensión que los que las generan.

Sin embargo, descubrimos posteriormente que los circuitos compartían la tierra de audio de la entrada directamente con la salida, por lo que se realizaba un cortocircuito directo entre la tierra de la alimentación (o alimentación negativa visto desde el amplificador de audio) y la tierra virtual, por lo que se cortocircuitaban 3.5 V. 

Esto era claro en el lado del MP3 ya que el cortocircuito era a través de un camino de baja impedancia y provocaba que se activara la protección, desactivando la salida de audio. Sin embargo, debido a la circuitería interna o a las conexiones de la radio, había una pequeña pero no mínima impedancia que provocaba un consumo muy elevado de corriente pero que no llegaba al amperio, por lo que la protección no se disparaba. Esto es muy peligroso ya que es una potencia que se está disipando en el interior del chip y probablemente provoque daños si se mantiene el circuito en dicha condición.

Para solucionar este problema, decidimos construir unos cables de sonido en los cuales solo se conecte el terminal que lleva la señal, deshaciendo la conexión que realizan los sensores internamente.

Con estos ajustes la radio funcionaba bien, sin demasiado problema (excepto el ruido del que hablaremos a continuación). Sin embargo, el MP3 presenta otro problema. Contraintuitivamente, a pesar de estar alimentado con una tensión unipolar, el circuito del MP3 consigue generar una tensión bipolar simétrica con la señal de audio, señal completamente incompatible con nuestro sistema de muestreo con un ADC de la placa.

Para mediar este problema, hemos montado otro circuito analógico accesorio que mediante un divisor de tensión simétrico y un condensador de desacoplo consigue añadir una componente continua de 1.65 V a la tensión simétrica de entrada, haciendola completamente compatible con el sistema. Se añade una resistencia de \textit{Pull-down} en la entrada para evitar picos de tensión en el encendido del sistema. Los valores de las resistencias pueden ser cualesquiera valores grandes siempre que sean iguales y el condensador interesa elegirlo lo más grande posible. El circuito está recogido en la \autoref{fig:2-6-sumador}. 

\begin{figure}[h]
    \centering
    \includegraphics[width=0.5\textwidth]{images/2/2-6/sumadorEsquematico.png}
    \caption{Esquemático del circuito sumador}
    \label{fig:2-6-sumador}
\end{figure}

Se ha montado el circuito sobre una placa de baquelita, como se puede ver en la imagen de la \autoref{fig:2-6-foto-sumador}.\ 

\begin{figure}[h]
    \centering
    \includegraphics[width=0.25\textwidth]{images/2/2-6/fotoSumador.jpg}
    \caption{Foto del circuito sumador analógico}
    \label{fig:2-6-foto-sumador}
\end{figure}

Hemos elegido un valor de $100\ k\Omega$ para las resistencias y aproximadamente $1\ \mu F$ para el condensador, lo cual nos ofrece buenos resultados. Sin embargo, primero realizamos las pruebas con un condensador cerámico y no conseguimos que la tensión se estabilizara, por lo que se desplazaba el valor lentamente hacia uno de los raíles de alimentación. La solución que encontramos fue sustituirlo por un condensador de tántalo, que si bien tiene un tamaño físico considerablemente superior, consigue mantener de manera muy estable la tensión del circuito.

Este circuito funciona correctamente pero tiene la desventaja de decrementar ligeramente la tensión de entrada, provocando una caída en la relación señal-ruido del sistema.

Una vez solucionado el problema de las masas, aparece un ruido muy elevado en forma de zumbido y ruido blanco, por lo que el audio es de bastante baja relacion calidad ruido. Esto es principalmente debido a la longitud de los conductores por lo que va la señal, la cantidad de circuitos que atraviesa, etcétera. 

Además, la desconexión de las masas de los cables de audio provoca que el camino de retorno de las señales tenga que atravesar el resto del circuito y se deja de tratar la señal como un par diferencial, por lo que se pierde bastante calidad debido a la diferencia de tensión en las masas y algún posible bucle de masa al que se le acople algún ruido.

Por todo ello, si se quiere disfrutar de la máxima calidad de audio que puede ofrecer nuestro circuito, se debe desconectar la masa de la placa de la del circuito de amplificación de audio. El principal inconveniente de ello es que se deja de poder gestionar el multiplexor y la habilitación a través de los GPIO y no se puede realizar el procesado digital de señales.

Si se conectan los pines de selección y habilitación a los raíles de alimentación para seleccionar la configuración y se conecta un jumper entre los pines de ADC y DAC, se tiene una buenísima calidad de sonido, tanto en el altavoz como en los cascos. Se puede igualmente controlar la radio y el MP3 mediante las interfaces ya que eso no depende de la masa del circuito.

La mejor solución a este problema sería la realización de un circuito con alimentación simétrica verdadera. Esto se podría conseguir mediante dos baterías en serie (aunque sería un desperdicio ya que una solo se utilizaría para la parte negativa de la señal y no alimentaría la placa) o mediante la generación de una tensión negativa con un convertidor, por ejemplo, de tipo reductor-elevador con topología inversora. Igualmente se tendría que tener en cuenta la naturaleza bipolar de la señal del MP3 y se debería incluir el circuito de \textit{offset} a la placa o utilizar un cambiador de nivel analógico.
\subsection{Conexión de la alimentación a la placa}

Otro problema significativo que hemos encontrado es la conexión de la placa a la alimentación por baterías. La placa \texttt{STM32F769NI-Disco} indica que se puede alimentar con una tensión de entre 7 y 12 voltios en el pin de \texttt{Vin} siempre que se seleccione \texttt{ext5V} en los jumpers de selección de alimentación.

Probamos a alimentarlo con una fuente de tensión de laboratorio y el circuito funcionaba adecuadamente, consumiendo aproximadamente $300\ mA$. Sin embargo, al alimentarlo desde nuestro subsistema analógico, la placa funciona adecuadamente durante aproximadamente cinco segundos para después quedarse congelado. Hemos podido comprobar que es únicamente la CPU que se congela ya que las DMAs de audio siguen funcionando, reproduciendo el último contenido del buffer en bucle.

Hemos comprobado que no es un problema de tensión ya que, aunque en la hoja de catálogo indique que la placa requiere de mínimo 7 voltios, funciona bien incluso con $6.5\ V$ de la fuente de alimentación.

Tampoco es un problema de corriente ya que, como se explica en el apartado de test hardware, la placa puede aportar mucha más corriente que el aproximadamente medio amperio que requiere la placa.

\section{Software}
\subsection{Interfaz de usuario}

Hemos desarrollado dos interfaces de usuario, una sobre una web y otra sobre una pantalla táctil.

\subsubsection{Interfaz web}
Para la interaccióncon el sistema, hemos diseñado un servidor web mediante el lenguaje \texttt{HTML}, estilizado mediante \texttt{CSS} y con algunas funciones creadas mediante JavaScript.

EL procesamiento de datos recibidos desde el servidor web y la creación dinámica de datos para la web se generan en diferentes archivos \texttt{CGI}, los cuales se explicarán en el \autoref{subsec:modulo-web}.

El servidor web cuenta con 4 páginas web diferentes, una principal, una para gestionar la radio, otra para gestionar el reproductor MP3 y otra para gestionar el procesado de audio. En todas estás páginas, en la esquina superior derecha, se encuentra la fecha y la hora del sistema. Todas estas páginas se explican a continuación.
\paragraph{Configuración General}
Esta página web cuenta con una sección llamada \textit{Camino de Audio}, la cual cuenta con 4 botones que nos permiten seleccionar tanto la entrada, Radio o MP3, tanto la salida de audio, Auriculares o Altavoz.

También cuenta con una sección llamada \textit{Bajo Consumo} que cuenta con tan solo un botón que nos permitirá poner el microcontrolador en modo bajo consumo.

Por último, cuenta con una sección llamada \textit{Consumo}, que contiene un widget, el cual se ha obtenido de \cite{JustGage} que nos permite visualizar de forma dinámica en consumo medido en el sistema.

\begin{figure}[h]
    \centering
    \includegraphics[width=0.8\textwidth]{images/3/3-1/3-1-1-1/Pagina_Principal.png}
    \caption{Página Principal}
    \label{fig:3-1-1-1-Principal}
\end{figure}
\paragraph{Página Radio}
Esta página web cuenta con una primera sección llamada \textit{Sintonizar una frecuencia} la cual nos permite introducir la frecuencia que deseemos sintonizar en el recuadro blanco. Luego, mediante el botón Sintonizar, podremos sintonizar dicha frecuencia en el Sintonizador FM.

A continuación, se encuentra una sección llamada \textit{Seek}, en la cual se encuentran dos botones. El primero, que contiene una flecha hacia arriba, nos permite realizar un \textit{SeekUp}, es decir, sintonizar una frecuancia mayor con más potencia que un determinado umbral. De forma análoga, el otro botón, ilustrado mediante una flecha hacia abajo, nos permite realizar un \textit{SeekDown}, es decir, sintonizar una frecuencia menor, pero que tenga más potencia que dicho determinado umbral.

La siguiente sección llamada \textit{Volumen}, contiene un slider horizontal que nos permite selecionar el volumen de la señal de audio. También encontramos un botón de mute, es decir, situar el volumen a 0.

Por último, encontramosla sección \textit{Salida}, la cual cuenta con dos botones que nos permiten seleccionar la salida de auido deseada entre Altavoz o Auriculares.

\begin{figure}[h]
    \centering
    \includegraphics[width=0.8\textwidth]{images/3/3-1/3-1-1-2/Pagina_Radio.png}
    \caption{Página Radio}
    \label{fig:3-1-1-2-Radio}
\end{figure}
\paragraph{Página MP3}
En primer lugar, nos econtramos con una sección llamada \textit{Canciones}, la cual cuenta con un menú desplegable con lista, con sus correspondientes nombres, de las posibles canciones. Junto a dicho menú, se cuentra un botón que nos permite confimar la canción seleccionada.

A continuación, encontramos la sección demoninada \textit{Control} la cual cuentra con 4 botones. El primero, nos permite seleccionar la canción anterior a la canción acutal. A continuación, nos encontramos con un botón que nos permite tanto pausar como continuar la reproducción de la canción actual. El siguiente botón nos permite seleccionar la siguiente canción de la lista. Por último, el botón de abajo, nos permite activar y desactivar la puesta en bucle de la canción actual.

De forma análoga a la página de la Radio, las siguientes dos secciones nos permiten modificar el volumen del sistema y la salida de la señal de audio.

\begin{figure}[H]
    \centering
    \includegraphics[width=0.8\textwidth]{images/3/3-1/3-1-1-3/Pagina_MP3.png}
    \caption{Página MP3}
    \label{fig:3-1-1-3-MP3}
\end{figure}
\paragraph{Procesamiento de Audio}

La última página web se encarga de todo el procesamiento de audio. En primer lugar, nos encontramos con la sección llamada \textit{Ecualizador}, el cual nos permite elegir, mediante unos sliders vertiales, entre un rango de valores, la ecualización que se desee aplicar en las diferentes bandas posibles.

A continuación, en la sección denominada \textit{Guardar Conf.}, nos encontramos un botón que nos permite guardar la configuración de los distintos filtros en la tarjeta microSD conectada al sistema.

Por último, de la misma manera que en las dos anteriores páginas, nos encontramos los controles que nos permiten modificar el volumen del sistema y seleccionar la salida de auido.

\begin{figure}[h]
    \centering
    \includegraphics[width=0.8\textwidth]{images/3/3-1/3-1-1-4/Pagina_Filtros.png}
    \caption{Página Filtros}
    \label{fig:3-1-1-4-Filtros}
\end{figure}
\subsection{Interfaz táctil}

\subsubsection{El alto nivel}
Una vez se ha configurado una función de bajo nivel para representar los píxeles en la pantalla, que se explica en el apartado \ref{}, se pueden utilizar las abstracciones de alto nivel que ofrece LVGL. 

La interfaz gráfica de RTAP consta de 4 pestañas: Inicio, Radio, MP3 y Filtros. A su vez, cada pestaña dispone de un lienzo en color gris claro, sobre el que se añaden paneles, con un fondo blanco y un borde gris de 2 píxeles de grosor. Cada panel implementa una funcionalidad, o muestra la información relevante. Todas las pestañas comparten una filosofía: Los paneles principales ocupan la totalidad de la pantalla, haciendo la interfaz muy intuitiva, y, si se desliza el panel hacia la parte superior, se muestran los créditos del proyecto: El título, la asignatura y los estudiantes involucrados. Además, todo el proyecto utiliza la fuente \textit{Montserrat}, variando el tamaño según la importancia del contenido. A continuación, se explica cada una de las pestañas, excluyendo el panel de créditos. 

\textbf{Inicio}

La pestaña de inicio consta de dos paneles principales: Configuración rápida y Consumo.
\begin{itemize}
    \item \textbf{Configuración rápida:} Este panel está formado por tres grupos de botones. El primer y el segundo grupo se emplean para seleccionar la salida y la entrada, respectivamente. Como el sistema no permite la opción de seleccionar las dos entradas o las dos salidas simultáneamente, estos botones son excluyentes: marcar uno desmarca el otro. Como estilo, se ha optado por un color azul claro para representar la opción seleccionada, y un color más oscuro para la no seleccionada. Por último, este panel tiene un botón que hace que el sistema entre en modo de bajo consumo. Los botones se distribuyen según una rejilla fija, y los grupos de estos se dividen mediante un espaciado uniforme, calculado de manera dinámica por LVGL.
    \item \textbf{Consumo:} Este panel está formado por una escala en forma de arco, que ocupa 225º de la circunferencia. Además, se le aplica una rotación de 180º para conseguir la posición deseada, que deja espacio suficiente para colocar una etiqueta de texto. 
    
    La escala, que se utiliza para representar de forma gráfica el consumo del sistema, admite valores entre 0 y 1500, que se interpretan como miliamperios. A la escala se le aplican tres formatos diferentes: El primero es un color verde, para los valores entre 0 y 375, el segundo, un color azul, entre 375 y 1125, y el tercero, en rojo, para los valores más altos, entre 1125 y 1500. En total, la escala cuenta con 25 \textit{ticks}, y cada 4 de estos, se introduce uno principal, es decir, para los valores más significativos, de 0 A, 0.25 A, 0.5 A, 0.75 A, 1 A , 1.25 A y 1.5 A. Estos \textit{ticks} principales tienen un grosor de 2 píxeles, mientras que los secundarios tienen un grosor de 1 píxel. LVGL calcula la longitud de cada \textit{tick} de manera dinámica. Por último, la escala cuenta con una aguja, en color rojo y con una longitud máxima de 80 píxeles, que apunta al valor del consumo instantáneo, que además se representa en la etiqueta de texto.

    El panel se organiza según una rejilla flexible organizada en forma de columna. Esto significa que simplemente se declaran los objetos y se les da un tamaño, y LVGL los coloca de forma dinámica apilados verticalmente. Sin embargo, para conseguir que la etiqueta se muestre en el lugar apropiado, se le da una posición fija relativa al panel.
\end{itemize}

\textbf{Radio}

La pestaña de radio está formada por tres paneles principales: Uno para la radio, otro para el volumen, y otro para la salida.

\begin{itemize}
    \item \textbf{Radio:} Este panel es el más grande de esta pestaña, ocupando dos terceras partes del espacio. Este panel es muy complejo, estando formado por una superposición de elementos. De manera simple: En la parte posterior encontramos un título, que indica que el panel es para sintonizar una frecuencia. En la posición inmediatamente inferior, se muestra un número, que en caso de no tener ningún valor, se mostrará en color sombreado. Hay varias formas de modificar este valor, que se detallan más adelante. Después encontramos una escala, que simula la apariencia de una radio clásica, y cuyos valores, entre 87 y 108, se interpretan como megahercios. Sin embargo, en un primer plano, encima de la escala, encontramos un slider, del cual solo se representa el mando, en color rojo y con un borde negro. En LVGL un slider solo puede tener valores enteros, por lo que este toma valores entre 870 y 1080, para mantener un decimal, interpretándose el valor como centenas de kilohercio. En la parte inferior hay un espacio en blanco, cuya utilidad se comenta a continuación, y después hay una fila con un objeto en forma de caja, inicialmente vacío, y a su derecha, tres botones.

    Empezando por el slider, que puede ser la opción más intuitiva, un deslizamiento sobre la escala tendrá diferentes consecuencias: El texto que tiene encima se modificará, acompañando el valor del texto, y, en caso de ser una cadena conocida, en el espacio en blanco que se mencionó anteriormente, se representará el texto correspondiente al nombre de la cadena. Sin embargo, para no saturar la cola de mensajes ni a la radio, la cadena marcada sólo se sintoniza una vez se ha soltado el slider.

    Otra posible forma de modificar el valor de la frecuencia es mediante los botones de \textit{Seek}. El sistema busca la siguiente cadena con una señal aceptable, la sintoniza, y modifica el valor del texto y del slider. Otra opción es tocar el valor del texto, y se desplegará un teclado numérico, que permite sintonizar de manera rápida y precisa una cadena, solo en caso de haber introducido un valor correcto. En caso de que el valor introducido no sea un valor posible, o sea un número mal formado, el slider se truncará hacia el valor correcto más próximo. Para cerrar el teclado, se puede utilizar cualquiera de los dos botones pensados para ello, o tocar en cualquier parte de la pantalla.

    Además, se dispone de un botón de favoritos, que almacenará en la caja todas las cadenas que se quieran guardar (no es persistente). En caso de que se conozca el nombre de la cadena, será esto lo que se muestre en la lista, en caso contrario, se mostrará únicamente la frecuencia. Para recuperar cualquier cadena guardada, solo habrá que buscarla en la lista.

    Por último, destacar que cualquier cambio en la web se verá reflejado de forma inmediata en el slider y en todos los cuadros de texto.
    \item \textbf{Volumen:} Este panel permite ajustar el volumen global del sistema. Se mantiene sincronizado con los paneles de volumen de otras pestañas, y presenta un botón de mute, que ejecuta una animación, variando el valor hasta 0 de forma suave, o, en caso de estar ya en cero, aumentándolo hasta la posición previa. Igual que con otros sliders, para no saturar los mecanismos de sincronización, el valor solo es enviado una vez se ha soltado.
    \item \textbf{Salida:} Este panel también permite seleccionar la salida del sistema. Se mantiene sincronizado con los botones que ofrecen la misma función en otras pestañas, y mantiene el mismo estilo que el de la pestaña de inicio para indicar cuál es la opción seleccionada.
\end{itemize}

\textbf{MP3}

La pestaña del MP3 mantiene muchas similitudes con la de la radio: Se compone de tres paneles, de los cuales el principal también ocupa dos tercios de la pantalla, y los dos paneles de su derecha son idénticos a los explicados anteriormente. 

El panel principal del MP3 también es un panel muy complejo. En este caso, el panel a su vez está compuesto por varios paneles. En un primer momento, se aprecian los 3 botones principales del MP3: anterior, play/pause y siguiente. El color de estos botones se ha pensado para romper con la monotonía del sistema, dando una impresión más alegre. El degradado se calcula automáticamente por LVGL, no se aplica ninguna textura que ocupe memoria.

En la parte inferior del panel hay una pestaña, que si se desliza hacia arriba, pasa a ocupar el 70\% del subpanel. Sin embargo, a los botones se les permite ocupar el 40\% del panel, con esto se logra la impresión de que los botones no terminan de esconderse nunca. Esta pestaña representa la lista de canciones que se tiene almacenada en la tarjeta SD. Al hacer click sobre una canción, esta se envía al MP3, y además, se reproduce una animación, que sustituye durante unos segundos al título de la pestaña, representando la canción. Nuevamente, si se oculta la lista de canciones, los botones de control pasan a ocupar el 100\% de su panel.

\textbf{Filtros}

La pestaña correspondiente a los filtros muestra las opciones de ecualización del sistema, además de algunos ajustes para el audio. En concreto, esta pestaña consta de 4 paneles: 
\begin{itemize}
    \item \textbf{Ecualizador:} Es el panel principal de la pestaña, y ocupa la mitad de esta. Se compone de 5 sliders, que manejan cada uno una banda. Como en sliders anteriores, el valor únicamente se envía cuando se libera el slider. Modificar un valor en la página web ejecuta una animación para sincronizar el valor.
    \item \textbf{Volumen:} Tiene un tamaño ligeramente inferior al de las pestañas anteriores, pero mantiene el mismo valor y funcionalidad.
    \item \textbf{Guardar config:} Es un pequeño panel con un único botón, para almacenar la configuración.
    \item \textbf{Configuración de audio:} Es parecido al panel de configuración rápida de la pestaña de inicio, manteniendo el estilo y todos los botones, excepto el de bajo consumo, que en este caso es reemplazado por un botón para restablecer el valor de los filtros. Pulsar el botón hace que se ejecuten 5 animaciones, una para cada slider, poniendo su valor a cero de forma suave.
\end{itemize}

\subsection{La parte técnica}
La placa STM32F769-disco que se ha utilizado en este proyecto dispone de una pantalla LCD táctil de 4.3", con una resolución de 800 x 480 píxeles y una profundidad del color de 16 bit. Esto permite crear interfaces gráficas atractivas, y, gracaias al uso de las DMA, con un coste computacional asumible. Por tanto, después de estudiar las diferentes opciones disponibles, se ha diseñado una interfaz completa, que permite el control completo del sistema.
\subsubsection{Software y librerías disponibles}
A la hora de crear una interfaz gráfica, la opción más lógica es utilizar una librería de más alto nivel, o un software de creación de interfaces, para abstraer el manejo de cada píxel individual, pero, en el mundo de los microcontroladores, estas opciones son bastante limitadas. Algunas de las opciones que se han estudiado son:
\begin{itemize}
  \item \textbf{Embedded Wizard:} Este software permite la creación de interfaces gráficas de una manera aparentemente sencilla, pero es código cerrado, y para utilizarlo de forma gratuita hay que asumir una marca de agua con su logo. Además, tienen su propio sistema operativo, que si bien no es muy distinto de las opciones conocidas, maximiza el riesgo de fallo a la hora de, por ejemplo, implementar el servidor web. Por tanto, esta opción se descartó.
  \item \textbf{EmWin:} Este software es la herramienta de Keil para la creación de interfaces, y viene incluida como \textit{software pack} dentro del programa. Incluye una función de generación de interfaces \textit{drag and drop}, lo que significa que, desde su programa, solo hay que colocar los elementos que se quieran tener en la interfaz en su sitio adecuado, todo desde una interfaz gráfica, abstrayendo el código. Sin embargo, las opciones de este software son muy limitadas, y las interfaces que genera tienen un aspecto rudimentario y obsoleto, por lo que esta opción también se descartó.
  \item \textbf{TouchGFX:} Este software, propiedad de STM, presenta una interfaz gráfica para la generación automática de código. Es un software muy potente, con el que es fácil generar interfaces modernas y visualmente atractivas. Además, incluye numerosos ejemplos, tanto en su aplicación, como en CubeMX. En un principio, se seleccionó esta opción, pero presenta el problema de ser de código cerrado, y es muy difícil adaptar el código que se genera para que sea compatible con Keil. Por tanto, finalmente se descartó.
  \item \textbf{LVGL:} Little Versatile Graphic Library es una librería de código abierto ampliamente utilizada en el mundo profesional. Varias empresas multinacionales, como Xiaomi o LG, utilizan actualmente adaptaciones de esta librería en algunos de sus productos. Es cierto que esta opción no dispone aún de herramientas para la generación automática de código, pero la librería es relativamente fácil de manejar. Con LVGL se pueden generar interfaces de todo tipo, y para el proyecto se ha seleccionado por su versatilidad y su fácil manejo. Además, ahora está presente en Keil como \textit{Software Pack}, por lo que su integración es absoluta. En nuestro proyecto, se utiliza la versión 9.1, que en el momento de redacción e implementación del proyecto, es la última versión estable de la librería.
\end{itemize}
\subsubsection{El bajo nivel}
LVGL es una librería de alto nivel, que es capaz de dibujar formas sobre un \textit{array}. Sin embargo, es responsabilidad de la persona que implementa la librería, el representar este \textit{array} en la pantalla. De este modo, el programador genera una función que representa un conjunto de píxeles en su pantalla, y LVGL se encarga de llamar a esta función cuando corresponda. Es decir, el tiempo de refresco de la pantalla variará según las necesidades del momento, liberando el uso de CPU cuando la pantalla tiene una imagen estática.

La primera aproximación posible para esta función es un simple bucle: itera todo el \textit{array} de píxeles y los envía a la pantalla. Sin embargo, esta aproximación es muy intensiva en el uso de CPU, además de poco eficiente. Por ello, una vez se ha determinado que el funcionamiento de la pantalla y el de LVGL es correcto, conviene modificarla. En el caso de RTAP, esta función utiliza la DMA 2, el \textit{Stream} 2 y el Canal 2, en configuración \textit{Memory To Memory}, consiguiendo alcanzar la tasa de refresco máxima que soporta la pantalla, de 30 fps, con un uso de la CPU mínimo, excepto cuando se ejecutan animaciones, que el uso de CPU aumenta considerablemente.

Una de las razones por las que la tasa de refresco disminuye es por el modo de funcionamiento de LVGL: admite diferentes configuraciones de \textit{buffer} que pasará como parámetro a la función de bajo nivel. En nuestro proyecto se utilizan de manera simultánea dos \textit{buffers} (la pantalla lee de uno mientras el controlador escribe en otro, y luego conmutan), pero cada uno solamente es del tamaño de la décima parte de la pantalla. Gracias a esta configuración es posible ahorrar mucha memoria, a cambio de un mayor uso de la CPU.

\subsubsection{Consideraciones sobre la pantalla y el bajo consumo}
La pantalla de la placa es un periférico complejo, que, como se ha explicado, utiliza la DMA para representar una imagen. Pero, cuando se entra en modo bajo consumo, la pantalla accede constantemente a los mismos datos: el reloj del procesador está parado, por lo que la información de los píxeles no se actualiza. Esto tiene un efecto indeseado, que es que la pantalla no se apaga, se queda con la última imagen que se cargó antes de entrar en modo de bajo consumo, lo que implica que el consumo es mayor del deseado. Por tanto hay que tomar la precaución de apagar la pantalla antes de entrar en este modo. RTAP implementa esta funcionalidad de manera transparente para el usuario: la aplicación apaga la pantalla cuando entra en modo de bajo consumo en cualquier caso, independientemente de que la orden se envíe desde el botón destinado a ello en la interfaz táctil o en la interfaz web.

\subsubsection{Consideraciones sobre LVGL}
\textbf{Sobre la memoria}

Después de haber repasado de forma general la interfaz del proyecto, se entiende que para representar un sistema tan complejo hace falta una cantidad considerable de memoria. En concreto, se asignan 131072 bytes de memoria a LVGL, y un stack de 12400 bytes al hilo que gestiona el funcionamiento de LVGL. La cantidad de memoria empleada es más que suficiente para el correcto funcionamiento de la placa, se deja un margen de en torno al 30\%. 

Para el proyecto de RTAP es necesario ser generosos con la cantidad de memoria de LVGL, porque, si bien es cierto que solo se mantiene en memoria los objetos que se están representando en el momento, hay diferentes elementos cuya cantidad de memoria variará entre dos usos de la aplicación. En concreto, estos elementos son: la lista de emisoras guardadas de la radio, y la lista de canciones del MP3. No se puede determinar con antelación el número de emisoras que almacenará el usuario, o el número de canciones que se deben representar. La configuración empleada está pensada para soportar cualquier caso de uso.

\textbf{Sobre los objetos}

En LVGL un objeto es un \textit{struct} con información sobre este. Los objetos en LVGL son extremadamente genéricos, hasta el punto de que, para la librería, un panel contenedor de botones es del mismo tipo que los propios botones, que a su vez son del mismo tipo que un teclado, que un \textit{slider} o que una escala. Esto implica una dificultad, y es que es que es muy sencillo pasar como parámetro un objeto equivocado a una función, con el correspondiente efecto indeseado, que suele ser un \textit{Hard Fault}, pero, también ofrece la ventaja del concepto de jerarquía: Independientemente del tipo de objeto, será hijo de otro objeto. El puntero al objeto padre se incluye en el \textit{struct} que define un objeto, además de punteros a todos los objetos hijos. El único tipo de objeto que no tiene padre es especial, y es la propia pantalla. Se pueden definir varias pantallas, pero en RTAP se trabaja únicamente con una, de modo que cualquier objeto es accesible para los demás a través de su árbol (esto evita el uso de variables globales).

\textbf{Sobre los eventos}

LVGL tiene diferentes métodos de notificación de eventos, muy útiles en un sistema orientado a eventos, como es una interfaz gráfica. Hay tres grandes grupos de eventos en nuestro proyecto: los generados por la pantalla, los generados por la web, y los generados periódicamente por el programa. En LVGL, un evento no es más que un \textit{struct} con la información pertinente, que incluye el objeto que lo provocó e información relevante, como el tipo de evento.

En cuanto a los primeros, desde la interfaz gráfica se asignan funciones de \textit{callback} a los objetos que las deben de generar, como los botones o \textit{sliders}. Como en LVGL un evento contiene la información sobre el tipo de evento, se pueden tomar numerosas decisiones. Por ejemplo, RTAP solo notifica de que un \textit{slider} ha variado su valor una vez se ha soltado, para no saturar a las colas de sincronización, pero, en el caso del \textit{slider} de la frecuencia de la radio, sí que se toma una acción mientras se está variando la posición de este: se representa el nombre de la cadena en el lugar específico. Por otro lado, desde la interfaz gráfica se pueden tomar acciones que afectarán a otros objetos del sistema: el botón de \textit{mute} (de cualquier pestaña) ejecuta una animación en el \textit{slider} del volumen, y el botón de restablecimiento de los valores de los filtros ejecuta animaciones en los \textit{sliders} de estos.

Todos los eventos se notifican a la web mediante elementos de sincronización, explicados en \ref{Error}. Cabe destacar que, desde el hilo que gestiona la pantalla y desde las diferentes funciones de \textit{callback}, no se toman acciones que afectan al sistema en general, únicamente se envían mensajes de control, para que un hilo controlador ejecute las acciones pertinentes.

Los eventos generados por la web presentan una dificultad: al ser la web un elemento invisible para LVGL, no se puede asignar una función de \textit{callback} directamente. Afortunadamente, la librería implementa métodos de notificación de eventos generados externamente, pero se debe de tener en cuenta otra consideración: el evento se notificará en el próximo refresco de la pantalla. Para cuidar la calidad del audio y evitar cortes, RTAP incluye un mecanismo que limita la tasa máxima de refresco de la pantalla. De este modo, cuando se está ejecutando una animación, por ejemplo, un cambio de pestaña, LVGL no ocupa toda la CPU, evitando así cortes molestos en el sonido. Pero esto implica que, como máximo, LVGL será notificado de un evento a los 15 milisegundos, siendo más común una notificación cada 33 milisegundos (30 refrescos por segundo). Si se genera más de un evento en este tiempo, se perderá. Por ello es importante que los \textit{sliders} de la web envíen también su valor únicamente cuando el usuario ha soltado el mismo, de otro modo, el comportamiento es impredecible. 

El último grupo de eventos es para los que no se provocan ni por la web ni por la pantalla, aunque, para LVGL, este tipo es exactamente igual que los de la web. Simplemente es algo que LVGL no puede predecir ni gestionar. En nuestro caso, el consumo entra dentro de este grupo, cuyo valor se mide cada segundo, y se notifica al hilo controlador de la pantalla, que a su vez lo representa en la interfaz.

\section{Depuración y test}
\subsection{Pruebas software}

\subsubsection{Módulo de procesado digital}

El módulo de procesado digital es el encargado de tomar las muestras, procesarlas y generar la señal de salida. Se utiliza un \texttt{ADC} de la placa para tomar muestras a $48\ kHz$ y un \texttt{DAC} para la generación de la salida, a la misma velocidad.

\paragraph{Temporización}
\label{para:temporizacion}

La temporización de este módulo es crítica, ya que es la que garantiza la calidad de la señal procesada. 

Se utiliza un \textit{Timer} de la placa, concretamente el \texttt{TIM2} para la generación de dicha señal. Para ello, el \textit{Timer} cuenta con una señal específica de sincronización para disparar eventos de conversión, \texttt{TRGO2}. Para habilitarla, se necesita además configurar el \textit{timer} como \textit{master}, como se puede ver en el \autoref{lst:conf-tim2}. Se utiliza un valor de \textit{Prescaler} de 9 y un \textit{Period} de 224, que junto con la velocidad de reloj del sistema de 216 MHz y el divisor de APB2 de 2, provocan una velocidad final de: 

\[
    f_s = \frac{f_{CLK}}{(PRESCALER + 1)(PERIOD + 1)\cdot Div_{APB2}} = \frac{216\cdot 10^6}{(9 + 1)(224 + 1)\cdot 2} = 48\ kHz
\]

\begin{lstlisting}[captionpos=t, caption={Configuración del TRGO del \textit{timer} de sincronización}]
TIM_ClockConfigTypeDef sClockConfig = {.ClockSource = TIM_CLOCKSOURCE_INTERNAL};
if (HAL_TIM_ConfigClockSource(&htim, &sClockConfig)) {
    return -1;
}

TIM_MasterConfigTypeDef sMasterConfig = {
    .MasterOutputTrigger = TIM_TRGO_UPDATE,
    .MasterSlaveMode = TIM_MASTERSLAVEMODE_DISABLE,
};
if (HAL_TIMEx_MasterConfigSynchronization(&htim, &sMasterConfig)) {
    return -1;
}
\end{lstlisting}

Una vez configurado este timer, habrá que configurar al \texttt{ADC} y al \texttt{DAC} para que realicen las conversiones disparados por dicha señal, como se verá en los siguientes apartados.

\paragraph{Adquisición de audio}

La adquisición de audio, como ya se ha comentado, se realiza a través de un \texttt{ADC}. Los parámetros más importantes de la configuración son:
\begin{enumerate}
    \item Adquisición a través del \texttt{GPIO A6}
    \item Uso del canal 6 de muestreo
    \item Resolución de 12 bits
    \item Trigger con el \texttt{TRGO} del \texttt{TIM2}
    \item Alineación de datos a la derecha
    \item Peticiones continuas a \texttt{DMA} habilitadas
\end{enumerate}

Además, se configura el canal 0 del \textit{Stream} 4 de la \texttt{DMA2} para este \texttt{ADC}. Esto significa que se utilizará este periférico para mover las muestras constantemente del \texttt{ADC} a la memoria sin necesitar ciclos del microcontrolador, lo cual alivia seriamente la carga que recae sobre el procesador. Esta \texttt{DMA} se arranca indicándole un \textit{buffer} de memoria e informa mediante una interrupción de cuando este está a la mitad de capacidad y cuando está completamente lleno. Se configura con los siguientes parámetros:
\begin{enumerate}
    \item Dirección de periférico a memoria
    \item Incremento de dirección de periférico deshabilitado
    \item Incremento de dirección de memoria habilitada
    \item Alineación de media palabra (16 bits)
    \item Modo circular
    \item FIFO desactivada
\end{enumerate} 

Se configura incrementando solo la dirección de memoria ya que el \texttt{ADC} solo tiene una salida (por tanto no tiene sentido incrementarlo) pero interesa que se llene la zona de memoria en lugar de sustituir constantemente el mismo dato. Además, el modo circular provoca que al llegar al final del buffer se vuelva a empezar, permitiendo un flujo constante de información. Esto es muy útil como se verá en los siguientes apartados.

\paragraph{Generación de señal}

La señal la generamos a través de un \texttt{DAC} de la placa. A pesar de que no se indica en la documentación de la placa, en el \textit{datasheet} del procesador vemos que el pin \texttt{PA4} está conectado al canal 1 del \texttt{DAC}, por lo que lo utilizamos para generar la señal de salida.

La configuración más significativa de dicho periférico es:
\begin{enumerate}
    \item Uso de \texttt{GPIO PA4}
    \item Trigger a partir del \texttt{TRGO} del \texttt{TIM2}
    \item Canal 1 de salida
\end{enumerate}

Se utiliza también una \texttt{DMA} para que el \texttt{DAC} tome muestras de una zona de memoria sin necesitar que el procesador se las configure. La configuración más relevante de dicho periférico es:
\begin{enumerate}
    \item Dirección de memoria a periférico
    \item Incremento de dirección de periférico deshabilitado
    \item Incremento de dirección de memoria habilitada
    \item Alineación de media palabra (16 bits)
    \item Modo circular
    \item FIFO desactivada
\end{enumerate} 

Esta \texttt{DMA} es de dirección de periférico a memoria, pero el resto de configuración es igual y por el mismo motivo que la del ADC.

\paragraph{Sistema de Doble Buffer}

Para realizar el procesamiento de audio, la implementación más directa es utilizar un \textit{buffer} de memoria y, cuando se llene gracias al \texttt{ADC}, realizar un procesamiento de su contenido y sacarlo a través del \texttt{DAC}. Sin embargo, esto provocaría que hubiera cortes constantes mientras se procesa el audio.

Para solucionarlo, la solución más común es el uso de un mecanismo de doble \textit{buffer} o mecanismo de \textit{ping-pong}. En este algoritmo, se tienen dos \textit{buffers} de memoria (o uno con el doble de tamaño como en nuestro caso). Mientras se llena uno de los dos, se toman las muestras de el otro y se procesan. Cuando se llena por completo, se comienza a llenar el otro y se procesan las muestras del primero. \cite{PingPongBuffer}

Para nuestro caso, necesitamos realizar esta construcción dos veces, una para el \texttt{ADC} y otra para el \texttt{DAC}. Por tanto, mientras una mitad se va llenando de muestras en un lado y reproduciendo en el otro, la otra mitad del \textit{buffer} del \texttt{ADC} se procesa y se coloca el resultado en la otra mitad del \texttt{DAC}.

\begin{figure}[h]
    \centering
    \includegraphics[width=0.5\textwidth]{images/3/3-2/DSP/doubleBuffer.png}
    \caption{Mecanismo de doble \textit{buffer} para audio}
    \label{fig:double-buffer}
\end{figure}

Este mecanismo nos permite reducir la carga del procesador sin perder calidad ni tener cortes notables en el audio.

Para implementarlo, se tiene un hilo de procesado de datos que está constantemente esperando a un par de \textit{flags} que se activan cuando se produce alguna de las interrupciones del \texttt{DMA} del \texttt{ADC}. Cuando estas se producen, se comprueba cual ha sido y se procesa la zona de memoria que corresponda.

\paragraph{Sistema de representación de coma fija}

Para el procesado de señal hemos utilizado la librería CMSIS DSP, la cual abstrae el procesamiento de señales y las matemáticas con muchos números en los procesadores de aquitectura ARM. \cite{CMSISDSPSoftware}

Para optimizarlo, permite la utilización de instrucciones \texttt{SIMD} las cuales realizan el mismo cálculo con muchos datos en un solo ciclo de reloj. Esto permite agilizar significativamente el procesado de grandes zonas de memoria, como pueden ser señales o vectores.

El principal problema de la librería son los formatos de representación con los que trabaja. No permite la utilización de números enteros normales, sino que obliga a utilizar números en coma flotante (más lentos y menos precisos) o en coma fija.

Para los números en coma fija, ARM utiliza el formato de representación de números \texttt{Q}. En este formato, los números se denotan como \texttt{QX.Y}, donde \texttt{X} es el número de bits que representan la parte entera y \texttt{Y} los bits que son parte decimal. Además, hay que indicar si el número es con signo o sin signo, si fuera con signo habría que añadir un bit que depende de la notación se debería incluir en la \texttt{X} o no (en el caso de ARM, la \texttt{X} sí incluye el bit de signo). Para manejar los números negativos se utiliza el complemento a dos.\cite{NumberFormat}

Concretamente en esta librería, se fija la parte entera a 1 bit (el de signo), por lo que el rango de números representables es aproximadamente $[-1, 1)$. Además, como siempre se utiliza el mismo número, directamente se omite, dando lugar a nombres como \texttt{Q31}, un número de 32 bits en los que el primero es el signo y los demás son parte decimal. % chktex 9

Resumiendo, en esta librería un número $x$ donde $b_n$ es el n-simo bit de menor peso, representado en \texttt{QN} es realmente:
\[ x = -b_{N} + \sum_{i = 0}^{N - 1} b_i\cdot 2^{i-N} \]

Se puede ver que este formato de representación es equivalente a dividir el formato de representación binario entre 2 elevado a el número de bits menos uno, que coincide con la $N$ del formato:

\[ x_{QN} = \frac{x_{bin}}{2^N} = \frac{-2^N\cdot b_N + \sum_{i=0}^{N - 1} b_i\cdot 2^i}{2^N} = -b_{N} + \sum_{i = 0}^{N - 1} b_i\cdot 2^{i-N} \]

Por tanto, se tiene un isomorfismo entre los dos formatos de representación que además es una operación lineal. Por tanto, todas las operaciones lineales se comportan bien con dicho isomorfismo, es decir, si se pasa un número de un formato a otro, se realiza una operación y se devuelve, es equivalente a hacer la operación en el formato original. Esto es así para las operaciones lineales que nos interesan, es decir, suma y multiplicación por una constante.

Como esas dos operaciones funcionan correctamente, se pueden realizar las siguientes operaciones utilizando números enteros de \texttt{N} bits como si fueran \texttt{QN}:
\begin{enumerate}
    \item Suma de números
    \item Producto de números
    \item Desplazamiento de números mientras no haya overflow (realmente equivalente a producto por potencia de dos)
    \item Convolución, al ser básicamente una suma ponderada
    \item Aplicación de filtro lineal, al poder reducirse a una convolución por la respuesta al impulso del filtro
\end{enumerate}

Por tanto, y como conclusión, se pueden utilizar las muestras de 12 bits del \texttt{ADC} que están almacenadas como números enteros de 16 bits como números \texttt{Q15} sin problema, por lo que la librería es ideal para nuestra aplicación. 

Para realizar pruebas y comprobar las propiedades de este formato de representación hemos utilizado una calculadora online que permite trabajar con números enteros, en coma fija y sus representaciones binarias. \footnote{\url{https://chummersone.github.io/qformat.html}}

\paragraph{Procesado Digital de la Señal}

Para el procesado de la señal, necesitamos un sistema que aplique ecualización de cinco bandas y permita ajustar el volumen del sistema. 

La ecualización se realiza mediante filtros bicuadrados, los cuales se caracterizan por configurarse con seis coeficientes: $(a_0, a_1, a_2, b_0, b_1, b_2)$. Con estos seis coeficientes se construye su función de transferencia discreta en el dominio de la transformada z:

\[ H(z) = \frac{b_0 + b_1\cdot z^{-1} + b_2\cdot z^{-2}}{a_0 + a_1\cdot z^{-1} + a_2\cdot z^{-2}}\]

Mediante filtros de esta forma se construyen filtros de tipo \textit{Peak EQ} o \textit{Shelf EQ}, los cuales incrementan o decrementan una banda de frecuencias dejando el resto sin alterar. La utilización de cinco de estos filtros en cascada permite la construcción de un ecualizador de cinco bandas. Para el cálculo de los coeficientes, hemos utilizado los que ofrece la propia librería en su documentación de referencia, específicamente para una aplicación equivalente a esta. \cite{GraphicAudioEqualizer}

Los coeficientes que ofrecen están en formato \texttt{Q31}, por lo que necesitamos utilizar filtros de dicha longitud. Esto no es un problema e incluso mejora la precisión intermedia de los cálculos. 

En CMSIS DSP, se utilizan filtros del tipo \texttt{arm\_biquad\_cas\_df1\_32x64\_q31} para los primeros dos filtros para tener más precisión en las bajas frecuencias y otros de tipo \texttt{arm\_biquad\_cascade\_df1\_q31} para el resto. Esto implica que cada banda cuenta realmente con dos etapas de filtros en cascada. Además, se fija el valor del coeficiente $a_0$ a $1$ para simplificar las cuentas. 

Por tanto, cada etapa necesita diez coeficientes en formato \texttt{q31} para su inicialización. Además, cada etapa necesita almacenar las variables de estado que se generan en cada iteración y que sirven para calcular la siguiente, por lo que se necesitan dos arrays de 8 elementos de tipo \texttt{q63} y tres arrays de 8 elementos de tipo \texttt{q31} adicionales.

Además, se necesitarán convertir las muestras de números \texttt{q15} a \texttt{q31} para realizar los cálculos, por lo que se tiene que crear un buffer intermedio del mismo tamaño que el de elementos a procesar.

La gestión de coeficientes es estática, es decir, no se puede cambiar un filtro una vez inicializado. Por ello, cuando se quiere cambiar la amplitud de una banda se necesita reinicializar el filtro. 

En total, se tienen diez coeficientes por banda y ganancia, cinco bandas y diecinueve valores posibles de ganancia por banda (desde -9 hasta 9 dB), por lo que se requieren 950 coeficientes distintos.

Además, como el formato de representación permite únicamente valores de módulo inferior a la unidad, se utiliza un \textit{shift} que multiplica todos los coeficientes por una potencia de dos.

Para el volumen se utiliza la función \texttt{arm\_scale\_q31}, que permite multiplicar toda una zona de memoria por un número, siempre que no sea máximo, que no se multiplica el número por nada. Como hemos explicado en el apartado anterior, para que la multiplicación en el dominio binario natural se mantenga, hay que multiplicar por la equivalencia en \texttt{Q31} de los números del $0$ al $1$ en pasos de $0.1$. Como adición, si el volumen es cero, se utiliza la función \texttt{arm\_fill\_q15} para rellenar el buffer con el número intermedio del rango (\texttt{2048}) y así reducir computaciones innecesarias que implican un consumo de potencia inútil.

Además, antes de procesar la señal se utiliza la función \texttt{arm\_offset\_q15} para restar el offset y añadirlo después de los cálculos y se utiliza \texttt{arm\_shift\_q31} para desplazar los datos y permitir así tener más margen de precisión en los cálculos. 

Finalmente, se eliminan los valores superiores a 4095 e inferiores a 0 ya que como solo se utilizan 12 de los 16 bits, estos valores provocan un overflow que cambia bruscamente la salida del \texttt{DAC}.

Se recoge la función de procesado en el siguiente fragmento de código.

\begin{lstlisting}[captionpos=t, caption={Función de procesado de audio}]
void processSamples(uint16_t* in, uint16_t* out) {
    if (volume == 0) {  // Volumen 0 -> No procesar salida (menos consumo)
        arm_fill_q15(2048, (q15_t*)out, DSP_BUFSIZE);
        return;
    }

    arm_offset_q15((q15_t*)in, -2048, (q15_t*)out, DSP_BUFSIZE);

    arm_q15_to_q31((q15_t*)out, midBuffer, DSP_BUFSIZE);

    arm_shift_q31(midBuffer, -DSP_SHIFT_MARGIN, midBuffer, DSP_BUFSIZE);

    if (volume < 10) {  // Aplicar volumen
        arm_scale_q31(midBuffer, volCoeffs[volume], 0, midBuffer, DSP_BUFSIZE);
    }

    // Aplicar filtros
    for (int i = 0; i < 2; i++) {
        arm_biquad_cas_df1_32x64_q31(&lowStageHandlers[i], midBuffer, midBuffer, DSP_BUFSIZE);
    }
    for (int i = 0; i < 3; i++) {
        arm_biquad_cascade_df1_q31(&highStageHandlers[i], midBuffer, midBuffer, DSP_BUFSIZE);
    }

    arm_shift_q31(midBuffer, DSP_SHIFT_MARGIN, midBuffer, DSP_BUFSIZE);


    arm_q31_to_q15(midBuffer, (q15_t*)out, DSP_BUFSIZE);

    // Recuperar offset
    arm_offset_q15((q15_t*)out, 2048, (q15_t*)out, DSP_BUFSIZE);

    // Saturar valores fuera de rango 
    arm_clip_q15((q15_t*)out, (q15_t*)out, 0, 4095, DSP_BUFSIZE);
}
\end{lstlisting}

\paragraph{Fichero de configuración}

Al igual que el módulo de control, la configuración de este módulo se puede ajustar mediante un fichero de cabecera, llamado \texttt{audioConfig.h}. En él, se pueden ajustar los pines utilizados en cada periférico, los canales utilizados y la temporización mediante los controles del timer.

Este fichero nos ha permitido utilizar el mismo código para realizar una primera implementación en la placa que cuenta con un procesador Cortex M4.
\subsection{Pruebas hardware}

\subsubsection{Pruebas de la placa de audio}

Para probar la placa de audio, la colocamos sola, alimentándola con una fuente de alimentación de laboratorio, colocando un Jumper entre el terminal de ADC y DAC para no tener en cuenta el procesado digital.

Después, se introduce una señal sinusoidal y se mide la amplitud de salida, calculando la respuesta en frecuencia del circuito. Se recoge una gráfica con la respuesta en frecuencia en decibelios relativos al máximo en la \autoref{fig:4-2-1-respuesta-cascos} (auriculares) y la \autoref{fig:4-2-1-respuesta-altavoces} (altavoces). Cabe destacar que la respuesta dibujada es en decibelios relativos al máximo, por lo que parece que tienen la misma amplitud, pero el altavoz duplica la amplitud de la señal de entrada.

\begin{figure}[h]
    \centering
    \includegraphics[width=0.5\textwidth]{images/4/4-2/respuesta-auriculares.png}
    \caption{Respuesta del circuito amplificador de auriculares}
    \label{fig:4-2-1-respuesta-cascos}
\end{figure}

\begin{figure}[h]
    \centering
    \includegraphics[width=0.5\textwidth]{images/4/4-2/respuesta-altavoces.png}
    \caption{Respuesta del circuito amplificador de altavoces}
    \label{fig:4-2-1-respuesta-altavoces}
\end{figure}

Comprobando el circuito de habilitación, funciona adecuadamente pero a veces se comporta de forma poco consistente al intentar desactivar el circuito cuando está habitado, pero generalmente se comporta correctamente. 

Por otro lado, conectamos distintas cargas y medimos el consumo del circuito, obteniendo los siguientes valores de consumo en función de la impedancia nominal del altavoz. Obtenemos los siguientes valores:
\begin{enumerate}
    \item Auriculates de $~40\ \Omega$: Consumo de $4\ mA$
    \item Altavoz de $8\ \Omega$: Consumo $64\ mA$
    \item Altavoz de $4\ \Omega$: Consumo de $164\ mA$
\end{enumerate}

Además, probando el cambiador de nivel, vemos que consigue una tensión de nivel bajo de $19.78\ mV$ y un $6.99\ V$.

\subsubsection{Pruebas del circuito de alimentación}

Para realizas las pruebas del circuito de alimentación, probamos primero a realizar una descarga y carga de la batería para probar este funcionamiento. Descargamos la batería hasta un valor aproximado de $3.5\ V$ y la volvimos a cargar con nuestro cargador hasta un valor de $4.05\ V$, en el cual la corriente de carga comienza a disminuir y el proceso de carga se ralentiza.

En condiciones normales, la batería carga con una corriente constante de entre $600$ y $700\ mA$, pero disminuye en el tramo final como se comentó en el diseño del circuito. También se ha probado a cargar el circuito a la vez que se carga la batería. Si la fuente tiene suficiente capacidad como para alimentar las dos cosas, hemos comprobado que así lo hace. Hemos probado por ejemplo con una carga de $10\ \Omega$, con lo que el consumo de aproximadamente $
700\ mA$ se suma a la carga obteniendo cerca de $1.5\ A$.

Por otro lado, caracterizamos la regulación de carga para comprobar que nuestro circuito mantuviera la tensión de salida incluso cuando es demandado una gran cantidad de corriente. Se ve que obtenemos un valor de $-39.012 V/A$, o $0.56\%/A$, un buen valor contando con que el consumo aproximado será de medio amperio. Se recoge la gráfica de las medidas en la \autoref{fig:4-2-2-regulacion-carga}.

\begin{figure}[h]
    \centering
    \includegraphics[width=0.5\textwidth]{images/4/4-2/regulacion-carga.png}
    \caption{Regulación de carga del circuito de baterías}
    \label{fig:4-2-2-regulacion-carga}
\end{figure}

Además, comprobamos que los indicadores luminosos funcionan correctamente ya que se enciende el verde cuando se conecta la alimentación y el rojo cuando se está cargando la batería. Al finalizar la carga, se apaga la luz roja y si se enchufa sin estar conectada parpadea el indicador rojo.

Mediante el osciloscopio medimos la tensión de salida del circuito de alimentación, observando que presenta un ruido en muy alta frecuencia, como se puede ver en la \autoref{fig:4-2-2-ruido}. Creemos que este pico de ruido es el culpable de la inestabilidad de los relojes de la placa, pero al colocar condensadores en paralelo no conseguimos reducirlo.

\begin{figure}[h]
    \centering
    \includegraphics[width=0.5\textwidth]{images/4/4-2/ruido.png}
    \caption{Ruido en la señal de alimentación}
    \label{fig:4-2-2-ruido}
\end{figure}
\section{Presupuesto final}
El presupuesto total del proyecto está recogido en la \autoref{tab:presupuesto}. Como se puede observar, tanto la tarjeta STM32F769NI como el sintonziador FM y el reproductor MP3, han sido cedidos por profesorado de la asignatura, por lo que su precio no se ve reflejado en el presupuesto total.

\setlength\LTleft{30pt}
\setlength\LTright{30pt}
\begin{longtable}{@{\extracolsep{\fill}}l r@{}}
    \toprule
        \textbf{Concepto} & \textbf{Precio} \\ \midrule
        Sensor NFC & 5.00\\
        Componentes Analógicos & 65.00 \\
        PCBs & 40.00 \\
        Batería & 8.00 \\
        Tarjeta STM32F769NI & Cedido \\
        Sintonziador FM & Cedido \\
        Reproductor MP3 & Cedido \\
        \midrule
        Total & 118.00\\
        \bottomrule
    \caption{Presupuesto del proyecto}
    \label{tab:presupuesto}
\end{longtable}
\section{Equipo de trabajo}
Trabajo individual:
	- Rubén Agustín Gonzalez: Interfaz de la pantalla de la placa, bajo consumo y uSD.
	- David Andrino Izquierdo: Esquemático de la alimentación, diseño de las PCBs, módulo de control, protector I2C y procesado digital de señal.
	- Estela Mora Barba: Esquemático del amplificador de audio, módulo NFC y uSD.
	- Fernando Sanz Giménez: Módulo RTC, radio, MP3 y web.

Hemos realizado de manera conjunta:
	- Integración del proyecto.
	- Realización de la memoria.
	- Powerpoints de las presentaciones.

A la hora de colaborar y juntar todas las partes del proyecto, hemos utilizado un repositorio en GitHub \footnote <\url{https://github.com/David-Andrino/ise-rtap }, con varias ramas creadas para cada miembro. Además, hemos estado trabajando tanto presencial como online, de forma tanto individual como colectiva. Ha habido mucha colaboración entre los miembros del grupo, no solo en lo referente al propio trabajo sino que también de forma emocional.
\include{files/7-Acronimos}

\nocite{*}
\printbibliography[title={Bibliografía}, heading=bibnumbered]

\begin{appendices}

% Anexo 1. Webench Report
\includepdf[pages=1,offset=0 0,scale=0.8, pagecommand={\section{Webench Design Report}\label{anexo:webench-report}}]{./files/8-Anexos/WebenchDesignReport.pdf}
\includepdf[pages=2-7,offset=0 0,scale=0.8, pagecommand={}]{./files/8-Anexos/WebenchDesignReport.pdf}

% Anexo 2. Circuito alimentacion
\includepdf[pages=1,offset=0 0,scale=0.8, pagecommand={\section{Circuito de alimentación}\label{anexo:circuito-alimentacion}}]{./files/8-Anexos/esquematicos.pdf}
\includepdf[pages=2-3,offset=0 0,scale=0.8, pagecommand={}]{./files/8-Anexos/esquematicos.pdf}

% Anexo 3. Circuito de audio
\includepdf[pages=4,offset=0 0,scale=0.8, pagecommand={\section{Circuito de alimentación}\label{anexo:circuito-audio}}]{./files/8-Anexos/esquematicos.pdf}

% Anexo 4. ThreadControl.h
\section{Estructura de los mensajes de control}
\label{anexo:mensajes-control}
\begin{lstlisting}[captionpos=t, caption={Fichero \texttt{controlThread.h} con las estructuras de mensajes}]
    /**
    * @file controlThread.h
    *
    * @brief Modulo de control de RTAP
    *
    * @author Ruben Agustin
    * @author David Andrino
    * @author Estela Mora
    * @author Fernando Sanz
    *
    * Modulo principal de inteligencia del sistema. Recibe eventos por 
    * una cola y reacciona acorde a ellos
    *
    */
    #ifndef CONTROL_THREAD_H
    #define CONTROL_THREAD_H

    #include <cmsis_os2.h>
    #include <stdint.h>

    #include "../main.h"
    #include "../SD/sd.h"

    /**
    * @brief Cola de mensajes de entrada al modulo
    */
    extern osMessageQueueId_t ctrl_in_queue;

    /**
    * @brief Inicializacion del modulo de control
    *
    * @return 0 si se ha realizado correctamente. Otro valor si no.
    */
    int Init_Control(sd_config_t* initial_config);

    // ==================================== MSG TYPES ======================================
    /**
    * @brief Enumeracion de los tipos de mensaje de entrada al modulo de control
    */
    typedef enum {
        MSG_NFC,   /**< Lectura de una tarjeta del NFC */
        MSG_LCD,   /**< Mensaje de entrada del LCD     */
        MSG_WEB,   /**< Mensaje de entrada de la web   */
        MSG_RTC,   /**< Mensaje de entrada del RTC     */
        MSG_CONS,  /**< Mensaje de entrada del consumo */
        MSG_RADIO, /**< Mensaje de entrada de la radio */
    } msg_ctrl_type_t;

    // ===================================== LCD ======================================
    /**
    * @brief Enumeracion de mensajes de entrada del LCD
    */
    typedef enum {
        LCD_VOL,        /**< Cambio de volumen. Contenido es el volumen */
        LCD_BANDS,      /**< Cambio de filtro. Contenido es primer byte la banda [0,4] segundo la cantidad [-9, 9] */
        LCD_RADIO_FREQ, /**< Cambio de frecuencia de la radio. Contenido es la frecuencia en centenas de kHz */
        LCD_SONG,       /**< Cambio de cancion. Contenido es el numero de cancion, empezando por la 0 */
        LCD_INPUT_SEL,  /**< Cambio de entrada. Contenido es 0 para la radio y 1 para MP3 */
        LCD_OUTPUT_SEL, /**< Cambio de salida. Contenido es 0 para cascos y 1 para altavoz */
        LCD_SAVE_SD,    /**< Guardar configuracion en la SD. Contenido ignorado */
        LCD_LOW_POWER,  /**< Entrar en modo bajo consumo. Contenido ignorado */
        LCD_LOOP,       /**< Poner cancion en bucle. Contenido ignorado*/
        LCD_SEEK,       /**< Hacer seek con la radio. Contenido es 0 para down y 1 para up */
        LCD_NEXT_SONG,  /**< Siguiente cancion */
        LCD_PREV_SONG,  /**< Anterior cancion */
        LCD_PLAY_PAUSE, /**< Alternar reproduccion de la cancion */
    } lcd_msg_type_t;

    /**
    * @brief Estructura para los mensajes de entrada del LCD
    */
    typedef struct {
        lcd_msg_type_t type;    /**< Tipo de mensaje del LCD */
        uint16_t       payload; /**< Contenido del mensaje. Depende del tipo. */
    } lcd_msg_t;

    // ==================================== WEB =======================================
    /**
    * @brief Enumeracion de mensajes de entrada de la web
    */
    typedef enum {
        WEB_INPUT_SEL,  /**< Cambio de entrada. Contenido es 0 para la radio y 1 para MP3 */
        WEB_OUTPUT_SEL, /**< Cambio de salida. Contenido es 0 para cascos y 1 para altavoz */
        WEB_LOW_POWER,  /**< Entrar en modo bajo consumo. Contenido ignorado */
        WEB_RADIO_FREQ, /**< Cambio de frecuencia de la radio. Contenido es la frecuencia en centenas de kHz */
        WEB_SEEK,       /**< Hacer seek con la radio. Contenido es 0 para down y 1 para up */
        WEB_VOL,        /**< Cambio de volumen. Contenido es el volumen */
        WEB_SONG,       /**< Cambio de cancion. Contenido es el numero de cancion */
        WEB_PLAY_PAUSE, /**< Alternar play y pause de la web */
        WEB_PREV_SONG,  /**< Anterior cancion */
        WEB_NEXT_SONG,  /**< Siguiente cancion*/
        WEB_BANDS,      /**< Cambio de filtro. Contenido es primer byte la banda [0,4] segundo la cantidad [-9, 9] */
        WEB_SAVE_SD,    /**< Guardar configuracion en la SD. Contenido ignorado */
        WEB_LOOP,       /**< Poner cancion en bucle. Contenido ignorado*/
    } web_msg_type_t;

    /**
    * @brief Estructura para los mensajes de entrada del LCD
    */
    typedef struct {
        web_msg_type_t type;    /**< Tipo del mensaje de */
        uint16_t       payload; /**< Contenido del mensaje. Depende del tipo */
    } web_msg_t;

    // ====================================== NFC ========================================
    /**
    * @brief Estructura para los mensajes de entrada del NFC
    */
    typedef struct {
        uint8_t  type;    /**< Tipo de mensaje. 0 para cancion y 1 para radio*/
        uint17_t content; /**< Numero de cancion o frecuencia en centenas*/
    } nfc_msg_t;

    // ====================================== RTC ========================================
    /**
    * @brief Estructura para los mensajes de entrada del RTC
    */
    typedef struct {
        uint8_t hour, minute, second, day, month, year;
    } rtc_msg_t;

    // ====================================== MSG ========================================
    /**
    * @brief Estructura para los mensajes de entrada
    */
    typedef struct {
        msg_ctrl_type_t type;    /**< Tipo de mensaje de entrada. 
                                    Dependiendo de este valor se debe interpretar el contenido */
        union {
            nfc_msg_t nfc_msg;   /**< Contenido de un mensaje de tipo MSG_NFC   */
            lcd_msg_t lcd_msg;   /**< Contenido de un mensaje de tipo MSG_LCD   */
            rtc_msg_t rtc_msg;   /**< Contenido de un mensaje de tipo MSG_RTC   */
            web_msg_t web_msg;   /**< Contenido de un mensaje de tipo MSG_WEB   */
            uint16_t  cons_msg;  /**< Contenido de un mensaje de tipo MSG_CONS  */
            uint32_t  radio_msg; /**< Contenido de un mensaje de tipo MSG_RADIO */
        };
    } msg_ctrl_t;

    #endif
    
\end{lstlisting}
\end{appendices}



\end{document}
