\subsection{Interfaz táctil}

La placa STM32F769-disco que se ha utilizado en este proyecto dispone de una pantalla LCD táctil de 4.3", con una resolución de 800 x 480 píxeles y una profundidad del color de 16 bit. Esto permite crear interfaces gráficas atractivas, y, gracaias al uso de las DMA, con un coste computacional asumible. Por tanto, después de estudiar las diferentes opciones disponibles, se ha diseñado una interfaz completa, que permite el control completo del sistema.

\subsubsection{Software y librerías disponibles}
A la hora de crear una interfaz gráfica, la opción más lógica es utilizar una librería de más alto nivel, o un software de creación de interfaces, para abstraer el manejo de cada píxel individual, pero, en el mundo de los microcontroladores, estas opciones son bastante limitadas. Algunas de las opciones que se han estudiado son:

\begin{itemize}
  \item Embedded Wizard: Este software permite la creación de interfaces gráficas de una manera aparentemente sencilla, pero es código cerrado, y para utilizarlo de forma gratuita hay que asumir una marca de agua con su logo. Además, tienen su propio sistema operativo, que si bien no es muy distinto de las opciones conocidas, máximiza el riesgo de fallo a la hora de, por ejemplo, implementar el servidor web. Por tanto, esta opción se descartó.
  \item Enwin: Este software es la herramienta de Keil para la creación de interfaces, y viene incluida como software pack dentro del programa. Incluye una función de generación de interfaces "drag and drop", lo que significa que, desde su programa, solo hay que colocar los elementos que se quieran tener en la interfaz en su sitio adecuado, todo desde una interfaz gráfica, abstrayéndo el código. Sin embargo, las opciones de este software son muy limitadas, y las interfaces que genera tienen un aspecto rudimentario y obsoleto, por lo que esta opción también se descartó.
  \item TouchGFX: Este software, propiedad de STM, presenta una interfaz gráfica para la generación automática de código. Es un software muy potente, con el que es fácil generar interfaces modernas y visualmente atractivas. Además, incluye numerosos ejemplos, tanto en su aplicación, como en CubeMX. En un principio, se seleccionó esta opción, pero presenta el problema de ser de código cerrado, y es muy difícil adaptar el código que se genera para que sea compatible con Keil. Por tanto, finalmente se descartó.
  \item LVGL: Little Versatile Graphic Library es una librería de código abierto ampliamente utilizada en el mundo profesional. Varias empresas multinacionales, como Xiaomi o LG, utilizan actualmente adaptaciones de esta librería en algunos de sus productos. Es cierto que esta opción no dispone aún de herramientas para la generación automática de código, pero la librería es relativamente fácil de manejar. Con LVGL se pueden generar interfaces de todo tipo, y para el proyecto se ha seleccionado esta opción por su versatilidad y su fácil manejo. Además, ahora está presente en Keil como Software Pack, por lo que su integración es absoluta.
\end{itemize}
