\subsection{Módulo Web}
\label{subsec:modulo-web}
Para la generación de las diferenctes páginas web del servidor y la creación de los diferentes scripts para interactuar con dichas páginas, hemos creados varios archivos con extension \texttt{CGI} como el que se adjunta a continuación:

\begin{lstlisting}[captionpos=b, caption={Ejemplo archivo .CGI}, language=html]
    t       <form action="index.cgi" method="post">
    c i 1       <input type="radio" id="entrada_radio" name="entrada" value="radio" OnClick="submit();" %s>
    t               <label for="entrada_radio" style="font-size: 20px;">Radio</label>
    c i 2       <input type="radio" id="entrada_mp3" name="entrada" value="mp3" OnClick="submit();" %s>
    t               <label for="entrada_mp3" style="font-size: 20px;">MP3</label>
    t       </form>
    t           <br><br>
    t       <form action="index.cgi" method="post">
    c i 3       <input type="radio" id="salida_altavoz" name="salida" value="altavoz" OnClick="submit();" %s>
    t               <label for="salida_altavoz" style="font-size: 20px;">Altavoz</label>
    c i 4       <input type="radio" id="salida_cascos" name="salida" value="cascos" OnClick="submit();" %s>
    t               <label for="salida_cascos" style="font-size: 20px;">Cascos</label>
    t       </form>
    \end{lstlisting}

Como se puede observar, existen dos tipos direfentes de líneas de código, las que empiezan por ``t'' y las que empiezan por ``c''. Las que empiezan por ``t'' son ignoradas por el compilador y no se procesan, en cambio, las que empiezan en  por ``c'' son atendidas y procesadas. A continuación, se comprueba la letra siguiente al espacio, en este caso la ``i'', debido a que nos encontramos en el archivo \textit{index.cgi}. Por último, se obtiene el siguiente carácter a continuación del espacio en blanco y, el propio programa del servidor web, tomará unas medidas u otras dependiendo de dicha letra o número. En todos los casos, se sustituirá el ``\%s'' que encontramos al final de dichas líneas de código por el conjunto de carácteres deseado mediante las función \textit{snprintf()}. A continuación se adjunta un trozo de código del programa principal del servidor web en el que se realiza dicha acción:

\begin{lstlisting}[captionpos=b, caption={Ejemplo procesamiento archivo .CGI}]
	switch(env[0]){
		case 'i':
			// Cases for index
			switch (env[2]){
				case '1':
				// Case for Radio Input
					len = sprintf (buf, &env[4], web_state.entrada == WEB_RADIO ? "checked" :"");
				break;
				case '2':
				// Case for MP3 Input
					len = sprintf (buf, &env[4], web_state.entrada == WEB_MP3 ? "checked" :"");
				break;
				case '3':
				// Case for Altavoz Output
					len = sprintf (buf, &env[4], web_state.salida == WEB_ALTAVOZ ? "checked" :"");
				break;
				case '4':
				// Case for Auriculares Output
					len = sprintf (buf, &env[4], web_state.salida == WEB_AURICULARES ? "checked" :"");
				break;
            }
        break;
    }
\end{lstlisting}

Por otra parte, para enviar los datos desde las páginas al programa principal del servidor, se ha optado por usar fótmularios con el método \textit{post}, los cuales se envían si se pulsa algún botón mediante la función \textit{OnClick="submit()"}.

Por último, vamos a comentar la función utilizada para actualizar periódicamente tanto la fecha y la hora como el consumo medido. Esto se ha realizado mediante una función creada en JavaScript llamada \textit{periodicUpdate()}. A continuación se adjunta dicha función en la página denomianada \textit{index}:

\begin{lstlisting}[captionpos=b, caption={Función updatePeriodic()}]
<script language=JavaScript type="text/javascript" src="xml_http.js"></script>
<script>
    var timeUpdate = new periodicObj("time.cgx", 500);
    function periodicUpdateRTC(){
      updateMultiple(timeUpdate,  plotRTCTime);
      rtc_elTime = setTimeout(periodicUpdateRTC, timeUpdate.period);
    }
    function plotRTCTime(){
      timeVal = document.getElementById("rtcTime").value;
      document.getElementById("rtc").textContent = timeVal;
      consumo = document.getElementById("cons_ref").value;
      jg1.refresh(consumo);
    }
</script>
\end{lstlisting}

Como se puede observar, dicha función consiste en una llamada, cada 500ms, al archivo \textit{time.cgx} de donde se obtienen periódicamente los valores de fecha y hora y de consumo. Cuando dichos valores son obtenido, se actualizan sus valores mostrados en las direfentes páginas web. A continuación se adjunta el código presente en el archivo \textit{time.cgx}, el cual tiene un comportamiento idéntico a los archivos con extension \texttt{.CGI} antes mencionados:

\begin{lstlisting}[captionpos=b, caption={Archivo time.cgx}]
t <form>
t <text>
t <id>rtcTime</id>
c h <value>%02d-%02d-20%02d %02d:%02d:%02d</value>
t </text>
t <text>
t <id>cons_ref</id>
c z <value>%04d</value>
t </text>
t </form>
.
\end{lstlisting}
