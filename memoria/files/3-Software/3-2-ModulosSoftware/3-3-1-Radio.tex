\subsubsection{Módulo Radio}
El módulo del Sintonizador FM consiste en un \textit{Thread} que se encarga de gestionar el funcionamiento del propio módulo. La comunicación entre el periférico y el microcontrolador se ha configuradao mediante el \textit{Driver I2C} proporcionado por \textit{CMSIS}.

Cuenta con dos colas de mensajes, una en la que el programa principal introduce los mensajes con los comandos que desea que ejecute el Sintonizador, como sintonizar una frecuencia, hacer un \textit{Seekup} o un \textit{SeekDown}, etc. La otra cola se utiliza para que la radio introduzca mensajes con información sobre las frecuencias sintonizadas. Los mensajes mandados por esta cola son del tipo \textit{MSG\_RADIO}.

También cuenta con un \textit{timer} periódico, con un periodo de medio segundo, que se encarga de leer los registros del sintonizador para comprobar su correcto funcionamiento.

Debido a que nuestro sistema cuenta con dos periféricos que utilizan el protocolo \texttt{I2C}, nos hemos visto obligados a utilizar un módulo de gestión del propio protocolo de comunicación, del cual se hablará en el \textbf{REFERENCIA AL APARTADO DEL I2C}.

El comportamiento básico de este módulo consiste en una espera mediante un \textit{osMessageQueueGet} de un mensaje proporcionado por el programa principal. Una vez el mensaje es recibido, es procesado y en función de su contenido, se ejecutará el comando correspondiente. En caso de ser un \textit{SeekUp} o un \textit{SeekDown}, el módulo del sintonizador informará al \textit{Thread} principal para que muestre la frecuencia en la que ha terminado.