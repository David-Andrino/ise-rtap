\section{Introducción}

\subsection{Especificaciones iniciales del sistema a diseñar y construir}

El proyecto consiste en el desarrollo de un sistema de procesamiento de audio en tiempo real con un sistema centrado en un microprocesador.

El sistema permitirá entrada de audio a través de un módulo de radio FM y la lectura de canciones de una tarjeta microSD externa. Tras capturar dicho audio, se le aplicará un procesamiento digital implementado mediante la librería CMSIS DSP y se reproducirá por un altavoz o unos auriculares.

La interacción con el sistema se realizará a través de una pantalla táctil o una interfaz web, la cual requiere conexión a través de Ethernet. Dicha interfaz permite seleccionar la entrada de audio, controlar el volumen y ecualización de audio y elegir la salida. Además, permite controlar la emisora de radio sintonizada.

El sistema almacenará los parámetros seleccionados (entrada, salida y filtros) y una lista de emisoras favoritas en la tarjeta microSD, los cuales se cargarán al iniciar el equipo.

El sistema será completamente autónomo, contando con una batería con su correspondiente circuitería de carga, protección y medición de consumo. Se ofrecerá información sobre la batería en la interfaz gráfica del sistema. Además, el sistema contará con un modo de bajo consumo para alargar la duración de dicha batería.

Para la selección de canciones se permitirá el uso de tarjetas NFC preconfiguradas con canciones o emisoras preconfiguradas, para la interacción sin interfaz gráfica. Por último, el sistema utilizará el RTC integrado en la placa para mantener la hora y el protocolo SNTP para la sincronización.

\subsection{Especificaciones finales del sistema diseñado y construido}

Las únicas especificaciones que no hemos logrado cumplir son:
\begin{enumerate}
    \item Funcionamiento completo con alimentación independientes. Más información en el \autoref{subsubsec:convertidor_elevador}
    \item Escritura de configuración en la tarjeta SD. Más información en el 
\end{enumerate}