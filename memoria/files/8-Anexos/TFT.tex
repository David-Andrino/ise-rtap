\begin{lstlisting}[captionpos=t, caption={Fichero \texttt{tft.h}}]
    /**
    * @file disp.h
    * 
    */
   
   #ifndef DISP_H
   #define DISP_H
   
   /*********************
    *      INCLUDES
    *********************/
   #include <stdint.h>
   #include "lvgl.h"
   
   /*********************
    *      DEFINES
    *********************/
   #define TFT_HOR_RES     800
   #define TFT_VER_RES     480
   #define TFT_NO_TEARING  0    /*1: no tearing but slower*/
   
   /**********************
    *      TYPEDEFS
    **********************/
   
   /**********************
    * GLOBAL PROTOTYPES
    **********************/
   void tft_init(void);
   /**********************
    *      MACROS
    **********************/
   
   #endif
   
\end{lstlisting}

\begin{lstlisting}[captionpos=t, caption={Fichero \texttt{tft.c}}]
    /**
    * @file disp.c
    *
    */
   
   /*********************
    *      INCLUDES
    *********************/
   #include "../../lv_conf.h"
   #include "lvgl.h"
   #include <string.h>
   
   #include "tft.h"
   #include "stm32f7xx_hal.h"
   
   #include "stm32f769i_discovery.h"
   #include "stm32f769i_discovery_lcd.h"
   #include "stm32f769i_discovery_sdram.h"
   
   /*********************
    *      DEFINES
    *********************/
   
   /* DMA Stream parameters definitions. You can modify these parameters to select
      a different DMA Stream and/or channel.
      But note that only DMA2 Streams are capable of Memory to Memory transfers. */
   #define DMA_STREAM               DMA2_Stream2
   #define DMA_CHANNEL              DMA_CHANNEL_2
   #define DMA_STREAM_IRQ           DMA2_Stream2_IRQn
   #define DMA_STREAM_IRQHANDLER    DMA2_Stream2_IRQHandler
   
   #if TFT_NO_TEARING
   #define ZONES               4       /*Divide the screen into zones to handle tearing effect*/
   #else
   #define ZONES               1
   #endif
   
   #define VSYNC               OTM8009A_800X480_VSYNC
   #define VBP                 OTM8009A_800X480_VBP
   #define VFP                 OTM8009A_800X480_VFP
   #define VACT                OTM8009A_800X480_HEIGHT
   #define HSYNC               OTM8009A_800X480_HSYNC
   #define HBP                 OTM8009A_800X480_HBP
   #define HFP                 OTM8009A_800X480_HFP
   #define HACT                (OTM8009A_800X480_WIDTH / ZONES)
   
   #define LAYER0_ADDRESS               (LCD_FB_START_ADDRESS)
   
   /**********************
    *      TYPEDEFS
    **********************/
   
   /**********************
    *  STATIC PROTOTYPES
    **********************/
   
   /*For LittlevGL*/
   static void tft_flush_cb(lv_disp_t * disp, const lv_area_t * area, uint8_t * pxmap);
   
   /*LCD*/
   static void LCD_Config(void);
   static void LTDC_Init(void);
   
   /*DMA to flush to frame buffer*/
   static void DMA_Config(void);
   static void DMA_TransferComplete(DMA_HandleTypeDef *han);
   static void DMA_TransferError(DMA_HandleTypeDef *han);
   
   /**********************
    *  STATIC VARIABLES
    **********************/
   
   extern LTDC_HandleTypeDef hltdc_discovery;
   extern DSI_HandleTypeDef hdsi_discovery;
   DSI_VidCfgTypeDef hdsivideo_handle;
   DSI_CmdCfgTypeDef CmdCfg;
   DSI_LPCmdTypeDef LPCmd;
   DSI_PLLInitTypeDef dsiPllInit;
   static RCC_PeriphCLKInitTypeDef  PeriphClkInitStruct;
   
   #if LV_COLOR_DEPTH == 16
   static uint16_t * my_fb = (uint16_t *)LAYER0_ADDRESS;
   #else
   static uint32_t * my_fb = (uint32_t *)LAYER0_ADDRESS;
   #endif
   
   lv_display_t * disp;
   
   static DMA_HandleTypeDef     DmaHandle;
   static volatile int32_t x1_flush;
   static volatile int32_t y1_flush;
   static volatile int32_t x2_flush;
   static volatile int32_t y2_flush;
   static volatile int32_t y_flush_act;
   static volatile const uint8_t * buf_to_flush;
   
   static volatile bool refr_qry;
   static volatile uint32_t t_last = 0;
   
   #if TFT_NO_TEARING
   uint8_t pPage[]       = {0x00, 0x00, 0x01, 0xDF}; /*   0 -> 479 */
   
   
   uint8_t pCols[ZONES][4] =
   {
   #if (ZONES == 4 )
     {0x00, 0x00, 0x00, 0xC7}, /*   0 -> 199 */
     {0x00, 0xC8, 0x01, 0x8F}, /* 200 -> 399 */
     {0x01, 0x90, 0x02, 0x57}, /* 400 -> 599 */
     {0x02, 0x58, 0x03, 0x1F}, /* 600 -> 799 */
   #elif (ZONES == 2 )
     {0x00, 0x00, 0x01, 0x8F}, /*   0 -> 399 */
     {0x01, 0x90, 0x03, 0x1F}
   #elif (ZONES == 1 )
     {0x00, 0x00, 0x03, 0x1F}, /*   0 -> 799 */
   #endif
   };
   #endif
   
   /**********************
    *      MACROS
    **********************/
   
   /**********************
    *   GLOBAL FUNCTIONS
    **********************/
   
   /**
    * Initialize your display here
    */
   void tft_init(void)
   {
       BSP_SDRAM_Init();
       /* Deactivate speculative/cache access to first FMC Bank to save FMC bandwidth */
       FMC_Bank1->BTCR[0] = 0x000030D2;
       LCD_Config();
   
       /* Send Display On DCS Command to display */
       HAL_DSI_ShortWrite(&(hdsi_discovery),
               0,
               DSI_DCS_SHORT_PKT_WRITE_P1,
               OTM8009A_CMD_DISPON,
               0x00);
   
       DMA_Config();
   
       static uint8_t buf1[TFT_HOR_RES * 48 * 2];
       static uint8_t buf2[TFT_HOR_RES * 48 * 2];
       disp = lv_display_create(800, 480);
       lv_display_set_buffers(disp, buf1, buf2, TFT_HOR_RES * 48 * 2, LV_DISP_RENDER_MODE_PARTIAL);
       lv_display_set_flush_cb(disp, tft_flush_cb);
   }
   
   /**********************
    *   STATIC FUNCTIONS
    **********************/
   
   static void tft_flush_cb(lv_disp_t * disp, const lv_area_t * area, uint8_t * pxmap)
   {
   
       SCB_CleanInvalidateDCache();
   
       /*Truncate the area to the screen*/
       int32_t act_x1 = area->x1 < 0 ? 0 : area->x1;
       int32_t act_y1 = area->y1 < 0 ? 0 : area->y1;
       int32_t act_x2 = area->x2 > TFT_HOR_RES - 1 ? TFT_HOR_RES - 1 : area->x2;
       int32_t act_y2 = area->y2 > TFT_VER_RES - 1 ? TFT_VER_RES - 1 : area->y2;
   
       x1_flush = act_x1;
       y1_flush = act_y1;
       x2_flush = act_x2;
       y2_flush = act_y2;
       y_flush_act = act_y1;
       buf_to_flush = pxmap;
   
       /*Use DMA instead of DMA2D to leave it free for GPU*/
       HAL_StatusTypeDef err;
       err = HAL_DMA_Start_IT(&DmaHandle,(uint32_t)buf_to_flush, (uint32_t)&my_fb[y_flush_act * TFT_HOR_RES + x1_flush],
                 (x2_flush - x1_flush + 1));
       if(err != HAL_OK)
       {
           while(1);	/*Halt on error*/
       }
   }
   
   static void LCD_Config(void)
   {
       DSI_PHY_TimerTypeDef  PhyTimings;
   
       /* Toggle Hardware Reset of the DSI LCD using
        * its XRES signal (active low) */
       BSP_LCD_Reset();
   
       /* Call first MSP Initialize only in case of first initialization
        * This will set IP blocks LTDC, DSI and DMA2D
        * - out of reset
        * - clocked
        * - NVIC IRQ related to IP blocks enabled
        */
       BSP_LCD_MspInit();
   
       /* LCD clock configuration */
       /* PLLSAI_VCO Input = HSE_VALUE/PLL_M = 1 Mhz */
       /* PLLSAI_VCO Output = PLLSAI_VCO Input * PLLSAIN = 417 Mhz */
       /* PLLLCDCLK = PLLSAI_VCO Output/PLLSAIR = 417 MHz / 5 = 83.4 MHz */
       /* LTDC clock frequency = PLLLCDCLK / LTDC_PLLSAI_DIVR_2 = 83.4 / 2 = 41.7 MHz */
       PeriphClkInitStruct.PeriphClockSelection = RCC_PERIPHCLK_LTDC;
       PeriphClkInitStruct.PLLSAI.PLLSAIN = 417;
       PeriphClkInitStruct.PLLSAI.PLLSAIR = 5;
       PeriphClkInitStruct.PLLSAIDivR = RCC_PLLSAIDIVR_2;
       HAL_RCCEx_PeriphCLKConfig(&PeriphClkInitStruct);
   
       /* Base address of DSI Host/Wrapper registers to be set before calling De-Init */
       hdsi_discovery.Instance = DSI;
   
       HAL_DSI_DeInit(&(hdsi_discovery));
   
       dsiPllInit.PLLNDIV  = 100;
       dsiPllInit.PLLIDF   = DSI_PLL_IN_DIV5;
       dsiPllInit.PLLODF   = DSI_PLL_OUT_DIV1;
   
       hdsi_discovery.Init.NumberOfLanes = DSI_TWO_DATA_LANES;
       hdsi_discovery.Init.TXEscapeCkdiv = 0x4;
       HAL_DSI_Init(&(hdsi_discovery), &(dsiPllInit));
   
       /* Configure the DSI for Command mode */
       CmdCfg.VirtualChannelID      = 0;
       CmdCfg.HSPolarity            = DSI_HSYNC_ACTIVE_HIGH;
       CmdCfg.VSPolarity            = DSI_VSYNC_ACTIVE_HIGH;
       CmdCfg.DEPolarity            = DSI_DATA_ENABLE_ACTIVE_HIGH;
   #if LV_COLOR_DEPTH == 16
       CmdCfg.ColorCoding           = DSI_RGB565;
   #else
       CmdCfg.ColorCoding           = DSI_RGB888;
   #endif
       CmdCfg.CommandSize           = HACT;
       CmdCfg.TearingEffectSource   = DSI_TE_EXTERNAL;
       CmdCfg.TearingEffectPolarity = DSI_TE_RISING_EDGE;
       CmdCfg.VSyncPol              = DSI_VSYNC_FALLING;
       CmdCfg.AutomaticRefresh      = DSI_AR_DISABLE;
       CmdCfg.TEAcknowledgeRequest  = DSI_TE_ACKNOWLEDGE_ENABLE;
       HAL_DSI_ConfigAdaptedCommandMode(&hdsi_discovery, &CmdCfg);
   
       LPCmd.LPGenShortWriteNoP    = DSI_LP_GSW0P_ENABLE;
       LPCmd.LPGenShortWriteOneP   = DSI_LP_GSW1P_ENABLE;
       LPCmd.LPGenShortWriteTwoP   = DSI_LP_GSW2P_ENABLE;
       LPCmd.LPGenShortReadNoP     = DSI_LP_GSR0P_ENABLE;
       LPCmd.LPGenShortReadOneP    = DSI_LP_GSR1P_ENABLE;
       LPCmd.LPGenShortReadTwoP    = DSI_LP_GSR2P_ENABLE;
       LPCmd.LPGenLongWrite        = DSI_LP_GLW_ENABLE;
       LPCmd.LPDcsShortWriteNoP    = DSI_LP_DSW0P_ENABLE;
       LPCmd.LPDcsShortWriteOneP   = DSI_LP_DSW1P_ENABLE;
       LPCmd.LPDcsShortReadNoP     = DSI_LP_DSR0P_ENABLE;
       LPCmd.LPDcsLongWrite        = DSI_LP_DLW_ENABLE;
       HAL_DSI_ConfigCommand(&hdsi_discovery, &LPCmd);
   
       /* Initialize LTDC */
       LTDC_Init();
   
       /* Start DSI */
       HAL_DSI_Start(&(hdsi_discovery));
   
       /* Configure DSI PHY HS2LP and LP2HS timings */
       PhyTimings.ClockLaneHS2LPTime = 35;
       PhyTimings.ClockLaneLP2HSTime = 35;
       PhyTimings.DataLaneHS2LPTime = 35;
       PhyTimings.DataLaneLP2HSTime = 35;
       PhyTimings.DataLaneMaxReadTime = 0;
       PhyTimings.StopWaitTime = 10;
       HAL_DSI_ConfigPhyTimer(&hdsi_discovery, &PhyTimings);
   
       /* Initialize the OTM8009A LCD Display IC Driver (KoD LCD IC Driver)
        *  depending on configuration set in 'hdsivideo_handle'.
        */
   #if LV_COLOR_DEPTH == 16
       OTM8009A_Init(OTM8009A_FORMAT_RBG565, LCD_ORIENTATION_LANDSCAPE);
   #else
       OTM8009A_Init(OTM8009A_FORMAT_RGB888, LCD_ORIENTATION_LANDSCAPE);
   #endif
       LPCmd.LPGenShortWriteNoP    = DSI_LP_GSW0P_DISABLE;
       LPCmd.LPGenShortWriteOneP   = DSI_LP_GSW1P_DISABLE;
       LPCmd.LPGenShortWriteTwoP   = DSI_LP_GSW2P_DISABLE;
       LPCmd.LPGenShortReadNoP     = DSI_LP_GSR0P_DISABLE;
       LPCmd.LPGenShortReadOneP    = DSI_LP_GSR1P_DISABLE;
       LPCmd.LPGenShortReadTwoP    = DSI_LP_GSR2P_DISABLE;
       LPCmd.LPGenLongWrite        = DSI_LP_GLW_DISABLE;
       LPCmd.LPDcsShortWriteNoP    = DSI_LP_DSW0P_DISABLE;
       LPCmd.LPDcsShortWriteOneP   = DSI_LP_DSW1P_DISABLE;
       LPCmd.LPDcsShortReadNoP     = DSI_LP_DSR0P_DISABLE;
       LPCmd.LPDcsLongWrite        = DSI_LP_DLW_DISABLE;
       HAL_DSI_ConfigCommand(&hdsi_discovery, &LPCmd);
   
       HAL_DSI_ConfigFlowControl(&hdsi_discovery, DSI_FLOW_CONTROL_BTA);
   
       /* Send Display Off DCS Command to display */
       HAL_DSI_ShortWrite(&(hdsi_discovery),
               0,
               DSI_DCS_SHORT_PKT_WRITE_P1,
               OTM8009A_CMD_DISPOFF,
               0x00);
   
   
   #if TFT_NO_TEARING
         HAL_DSI_LongWrite(&hdsi_discovery, 0, DSI_DCS_LONG_PKT_WRITE, 4, OTM8009A_CMD_CASET, pCols[0]);
         HAL_DSI_LongWrite(&hdsi_discovery, 0, DSI_DCS_LONG_PKT_WRITE, 4, OTM8009A_CMD_PASET, pPage);
   
       /* Enable GPIOJ clock */
       __HAL_RCC_GPIOJ_CLK_ENABLE();
   
       /* Configure DSI_TE pin from MB1166 : Tearing effect on separated GPIO from KoD LCD */
       /* that is mapped on GPIOJ2 as alternate DSI function (DSI_TE)                      */
       /* This pin is used only when the LCD and DSI link is configured in command mode    */
       /* Not used in DSI Video mode.
        */
       GPIO_InitTypeDef  GPIO_Init_Structure;
       GPIO_Init_Structure.Pin       = GPIO_PIN_2;
       GPIO_Init_Structure.Mode      = GPIO_MODE_AF_PP;
       GPIO_Init_Structure.Pull      = GPIO_NOPULL;
       GPIO_Init_Structure.Speed     = GPIO_SPEED_HIGH;
       GPIO_Init_Structure.Alternate = GPIO_AF13_DSI;
       HAL_GPIO_Init(GPIOJ, &GPIO_Init_Structure);
   
       static uint8_t ScanLineParams[2];
   #if ZONES == 2
       uint16_t scanline = 200;
   #elif ZONES == 4
       uint16_t scanline = 283;
   #endif
       ScanLineParams[0] = scanline >> 8;
       ScanLineParams[1] = scanline & 0x00FF;
   
       HAL_DSI_LongWrite(&hdsi_discovery, 0, DSI_DCS_LONG_PKT_WRITE, 2, 0x44, ScanLineParams);
       /* set_tear_on */
       HAL_DSI_ShortWrite(&hdsi_discovery, 0, DSI_DCS_SHORT_PKT_WRITE_P1, OTM8009A_CMD_TEEON, 0x00);
   #endif
   
   }
   
   #if TFT_NO_TEARING
   /**
   * LCD_SetUpdateRegion
   */
   void LCD_SetUpdateRegion(int idx)
   {
     HAL_DSI_LongWrite(&hdsi_discovery, 0, DSI_DCS_LONG_PKT_WRITE, 4, OTM8009A_CMD_CASET, pCols[idx]);
   }
   #endif
   /**
    * @brief
    * @param  None
    * @retval None
    */
   static void LTDC_Init(void)
   {
       /* DeInit */
       HAL_LTDC_DeInit(&hltdc_discovery);
   
       /* LTDC Config */
       /* Timing and polarity */
       hltdc_discovery.Init.HorizontalSync = HSYNC;
       hltdc_discovery.Init.VerticalSync = VSYNC;
       hltdc_discovery.Init.AccumulatedHBP = HSYNC+HBP;
       hltdc_discovery.Init.AccumulatedVBP = VSYNC+VBP;
       hltdc_discovery.Init.AccumulatedActiveH = VSYNC+VBP+VACT;
       hltdc_discovery.Init.AccumulatedActiveW = HSYNC+HBP+HACT;
       hltdc_discovery.Init.TotalHeigh = VSYNC+VBP+VACT+VFP;
       hltdc_discovery.Init.TotalWidth = HSYNC+HBP+HACT+HFP;
   
       /* background value */
       hltdc_discovery.Init.Backcolor.Blue = 0;
       hltdc_discovery.Init.Backcolor.Green = 0;
       hltdc_discovery.Init.Backcolor.Red = 0;
   
       /* Polarity */
       hltdc_discovery.Init.HSPolarity = LTDC_HSPOLARITY_AL;
       hltdc_discovery.Init.VSPolarity = LTDC_VSPOLARITY_AL;
       hltdc_discovery.Init.DEPolarity = LTDC_DEPOLARITY_AL;
       hltdc_discovery.Init.PCPolarity = LTDC_PCPOLARITY_IPC;
       hltdc_discovery.Instance = LTDC;
   
       HAL_LTDC_Init(&hltdc_discovery);
   
   
       LCD_LayerCfgTypeDef  Layercfg;
   
      /* Layer Init */
      Layercfg.WindowX0 = 0;
      Layercfg.WindowX1 = HACT;
      Layercfg.WindowY0 = 0;
      Layercfg.WindowY1 = BSP_LCD_GetYSize();
   #if LV_COLOR_DEPTH == 16
      Layercfg.PixelFormat = LTDC_PIXEL_FORMAT_RGB565;
   #else
      Layercfg.PixelFormat = LTDC_PIXEL_FORMAT_ARGB8888;
   #endif
      Layercfg.FBStartAdress = LAYER0_ADDRESS;
      Layercfg.Alpha = 255;
      Layercfg.Alpha0 = 0;
      Layercfg.Backcolor.Blue = 0;
      Layercfg.Backcolor.Green = 0;
      Layercfg.Backcolor.Red = 0;
      Layercfg.BlendingFactor1 = LTDC_BLENDING_FACTOR1_PAxCA;
      Layercfg.BlendingFactor2 = LTDC_BLENDING_FACTOR2_PAxCA;
      Layercfg.ImageWidth = BSP_LCD_GetXSize();;
      Layercfg.ImageHeight = BSP_LCD_GetYSize();
   
      HAL_LTDC_ConfigLayer(&hltdc_discovery, &Layercfg, 0);
   
   }
   
   #if TFT_NO_TEARING
   static volatile uint32_t LCD_ActiveRegion;
   
   /**
     * @brief  Tearing Effect DSI callback.
     * @param  hdsi: pointer to a DSI_HandleTypeDef structure that contains
     *               the configuration information for the DSI.
     * @retval None
     */
   void HAL_DSI_TearingEffectCallback(DSI_HandleTypeDef *hdsi)
   {
       if(refr_qry) {
           LCD_ActiveRegion = 1;
           HAL_DSI_Refresh(hdsi);
           refr_qry = false;
       }
   }
   
   void HAL_DSI_EndOfRefreshCallback(DSI_HandleTypeDef *hdsi)
   {
   
       if(LCD_ActiveRegion < ZONES )
       {
           /* Disable DSI Wrapper */
           __HAL_DSI_WRAPPER_DISABLE(hdsi);
           /* Update LTDC configuaration */
           LTDC_LAYER(&hltdc_discovery, 0)->CFBAR  = LAYER0_ADDRESS + LCD_ActiveRegion  * HACT * 2;
           __HAL_LTDC_RELOAD_CONFIG(&hltdc_discovery);
           __HAL_DSI_WRAPPER_ENABLE(hdsi);
   
           LCD_SetUpdateRegion(LCD_ActiveRegion++);
           /* Refresh the right part of the display */
           HAL_DSI_Refresh(hdsi);
   
       }
       else
       {
           __HAL_DSI_WRAPPER_DISABLE(&hdsi_discovery);
           LTDC_LAYER(&hltdc_discovery, 0)->CFBAR  = LAYER0_ADDRESS;
   
           __HAL_LTDC_RELOAD_CONFIG(&hltdc_discovery);
           __HAL_DSI_WRAPPER_ENABLE(&hdsi_discovery);
   
           LCD_SetUpdateRegion(0);
           lv_disp_flush_ready(disp);
       }
   }
   #endif
   
   /**
     * @brief  Configure the DMA controller according to the Stream parameters
     *         defined in main.h file
     * @note  This function is used to :
     *        -1- Enable DMA2 clock
     *        -2- Select the DMA functional Parameters
     *        -3- Select the DMA instance to be used for the transfer
     *        -4- Select Callbacks functions called after Transfer complete and
                  Transfer error interrupt detection
     *        -5- Initialize the DMA stream
     *        -6- Configure NVIC for DMA transfer complete/error interrupts
     * @param  None
     * @retval None
     */
   static void DMA_Config(void)
   {
     /*## -1- Enable DMA2 clock #################################################*/
     __HAL_RCC_DMA2_CLK_ENABLE();
   
     /*##-2- Select the DMA functional Parameters ###############################*/
     DmaHandle.Init.Channel = DMA_CHANNEL;                     /* DMA_CHANNEL_2                    */
     DmaHandle.Init.Direction = DMA_MEMORY_TO_MEMORY;          /* M2M transfer mode                */
     DmaHandle.Init.PeriphInc = DMA_PINC_ENABLE;               /* Peripheral increment mode Enable */
     DmaHandle.Init.MemInc = DMA_MINC_ENABLE;                  /* Memory increment mode Enable     */
   #if LV_COLOR_DEPTH == 16
     DmaHandle.Init.PeriphDataAlignment = DMA_PDATAALIGN_HALFWORD; /* Peripheral data alignment : 16bit */
     DmaHandle.Init.MemDataAlignment = DMA_MDATAALIGN_HALFWORD;    /* memory data alignment : 16bit     */
   #else
     DmaHandle.Init.PeriphDataAlignment = DMA_PDATAALIGN_WORD; /* Peripheral data alignment : 16bit */
     DmaHandle.Init.MemDataAlignment = DMA_MDATAALIGN_WORD;    /* memory data alignment : 16bit     */
   #endif
     DmaHandle.Init.Mode = DMA_NORMAL;                         /* Normal DMA mode                  */
     DmaHandle.Init.Priority = DMA_PRIORITY_HIGH;              /* priority level : high            */
     DmaHandle.Init.FIFOMode = DMA_FIFOMODE_ENABLE;            /* FIFO mode enabled                */
     DmaHandle.Init.FIFOThreshold = DMA_FIFO_THRESHOLD_1QUARTERFULL; /* FIFO threshold: 1/4 full   */
     DmaHandle.Init.MemBurst = DMA_MBURST_SINGLE;              /* Memory burst                     */
     DmaHandle.Init.PeriphBurst = DMA_PBURST_SINGLE;           /* Peripheral burst                 */
   
     /*##-3- Select the DMA instance to be used for the transfer : DMA2_Stream2 #*/
     DmaHandle.Instance = DMA_STREAM;
   
     /*##-4- Initialize the DMA stream ##########################################*/
     if(HAL_DMA_Init(&DmaHandle) != HAL_OK)
     {
       while(1);
     }
   
     /*##-5- Select Callbacks functions called after Transfer complete and Transfer error */
     HAL_DMA_RegisterCallback(&DmaHandle, HAL_DMA_XFER_CPLT_CB_ID, DMA_TransferComplete);
     HAL_DMA_RegisterCallback(&DmaHandle, HAL_DMA_XFER_ERROR_CB_ID, DMA_TransferError);
   
     /*##-6- Configure NVIC for DMA transfer complete/error interrupts ##########*/
     HAL_NVIC_SetPriority(DMA_STREAM_IRQ, 0, 0);
     HAL_NVIC_EnableIRQ(DMA_STREAM_IRQ);
   }
   
   /**
     * @brief  DMA conversion complete callback
     * @note   This function is executed when the transfer complete interrupt
     *         is generated
     * @retval None
     */
   static void DMA_TransferComplete(DMA_HandleTypeDef *han)
   {
       y_flush_act ++;
   
       if(y_flush_act > y2_flush) {
   #if TFT_NO_TEARING
           if(lv_disp_flush_is_last(disp)) refr_qry = true;
           else lv_disp_flush_ready(disp);
   #else
           if(lv_disp_flush_is_last(disp)) HAL_DSI_Refresh(&hdsi_discovery);
   
           lv_disp_flush_ready(disp);
   #endif
       } else {
         buf_to_flush += (x2_flush - x1_flush + 1) * 2;
         /*##-7- Start the DMA transfer using the interrupt mode ####################*/
         /* Configure the source, destination and buffer size DMA fields and Start DMA Stream transfer */
         /* Enable All the DMA interrupts */
         if(HAL_DMA_Start_IT(han,(uint32_t)buf_to_flush, (uint32_t)&my_fb[y_flush_act * TFT_HOR_RES + x1_flush],
                             (x2_flush - x1_flush + 1)) != HAL_OK)
         {
           while(1);	/*Halt on error*/
         }
       }
   }
   
   /**
     * @brief  DMA conversion error callback
     * @note   This function is executed when the transfer error interrupt
     *         is generated during DMA transfer
     * @retval None
     */
   static void DMA_TransferError(DMA_HandleTypeDef *han)
   {
       while(1);
   }
   
   
   /**
     * @brief  This function handles DMA Stream interrupt request.
     * @param  None
     * @retval None
     */
   void DMA_STREAM_IRQHANDLER(void)
   {
       /* Check the interrupt and clear flag */
       HAL_DMA_IRQHandler(&DmaHandle);
   }
   
   /**
     * @brief  Initialize the BSP LCD Msp.
     * Do not DMA2D is initialized by LVGL
     */
   void BSP_LCD_MspInit(void)
   {
     /** @brief Enable the LTDC clock */
     __HAL_RCC_LTDC_CLK_ENABLE();
   
     /** @brief Toggle Sw reset of LTDC IP */
     __HAL_RCC_LTDC_FORCE_RESET();
     __HAL_RCC_LTDC_RELEASE_RESET();
   
     /** @brief Enable DSI Host and wrapper clocks */
     __HAL_RCC_DSI_CLK_ENABLE();
   
     /** @brief Soft Reset the DSI Host and wrapper */
     __HAL_RCC_DSI_FORCE_RESET();
     __HAL_RCC_DSI_RELEASE_RESET();
   
     /** @brief NVIC configuration for LTDC interrupt that is now enabled */
     HAL_NVIC_SetPriority(LTDC_IRQn, 3, 0);
     HAL_NVIC_EnableIRQ(LTDC_IRQn);
   
     /** @brief NVIC configuration for DSI interrupt that is now enabled */
     HAL_NVIC_SetPriority(DSI_IRQn, 3, 0);
     HAL_NVIC_EnableIRQ(DSI_IRQn);
   }
   
   void DSI_IRQHandler(void){
     HAL_DSI_IRQHandler(&hdsi_discovery);
   }   
\end{lstlisting}