\label{anexo:mensajes-control}
\begin{lstlisting}[captionpos=t, caption={Fichero \texttt{controlThread.h} con las estructuras de mensajes}]
    /**
    * @file controlThread.h
    *
    * @brief Modulo de control de RTAP
    *
    * @author Ruben Agustin
    * @author David Andrino
    * @author Estela Mora
    * @author Fernando Sanz
    *
    * Modulo principal de inteligencia del sistema. Recibe eventos por 
    * una cola y reacciona acorde a ellos
    *
    */
    #ifndef CONTROL_THREAD_H
    #define CONTROL_THREAD_H

    #include <cmsis_os2.h>
    #include <stdint.h>

    #include "../main.h"
    #include "../SD/sd.h"

    /**
    * @brief Cola de mensajes de entrada al modulo
    */
    extern osMessageQueueId_t ctrl_in_queue;

    /**
    * @brief Inicializacion del modulo de control
    *
    * @return 0 si se ha realizado correctamente. Otro valor si no.
    */
    int Init_Control(sd_config_t* initial_config);

    // ==================================== MSG TYPES ======================================
    /**
    * @brief Enumeracion de los tipos de mensaje de entrada al modulo de control
    */
    typedef enum {
        MSG_NFC,   /**< Lectura de una tarjeta del NFC */
        MSG_LCD,   /**< Mensaje de entrada del LCD     */
        MSG_WEB,   /**< Mensaje de entrada de la web   */
        MSG_RTC,   /**< Mensaje de entrada del RTC     */
        MSG_CONS,  /**< Mensaje de entrada del consumo */
        MSG_RADIO, /**< Mensaje de entrada de la radio */
    } msg_ctrl_type_t;

    // ===================================== LCD ======================================
    /**
    * @brief Enumeracion de mensajes de entrada del LCD
    */
    typedef enum {
        LCD_VOL,        /**< Cambio de volumen. Contenido es el volumen */
        LCD_BANDS,      /**< Cambio de filtro. Contenido es primer byte la banda [0,4] segundo la cantidad [-9, 9] */
        LCD_RADIO_FREQ, /**< Cambio de frecuencia de la radio. Contenido es la frecuencia en centenas de kHz */
        LCD_SONG,       /**< Cambio de cancion. Contenido es el numero de cancion, empezando por la 0 */
        LCD_INPUT_SEL,  /**< Cambio de entrada. Contenido es 0 para la radio y 1 para MP3 */
        LCD_OUTPUT_SEL, /**< Cambio de salida. Contenido es 0 para cascos y 1 para altavoz */
        LCD_SAVE_SD,    /**< Guardar configuracion en la SD. Contenido ignorado */
        LCD_LOW_POWER,  /**< Entrar en modo bajo consumo. Contenido ignorado */
        LCD_LOOP,       /**< Poner cancion en bucle. Contenido ignorado*/
        LCD_SEEK,       /**< Hacer seek con la radio. Contenido es 0 para down y 1 para up */
        LCD_NEXT_SONG,  /**< Siguiente cancion */
        LCD_PREV_SONG,  /**< Anterior cancion */
        LCD_PLAY_PAUSE, /**< Alternar reproduccion de la cancion */
    } lcd_msg_type_t;

    /**
    * @brief Estructura para los mensajes de entrada del LCD
    */
    typedef struct {
        lcd_msg_type_t type;    /**< Tipo de mensaje del LCD */
        uint16_t       payload; /**< Contenido del mensaje. Depende del tipo. */
    } lcd_msg_t;

    // ==================================== WEB =======================================
    /**
    * @brief Enumeracion de mensajes de entrada de la web
    */
    typedef enum {
        WEB_INPUT_SEL,  /**< Cambio de entrada. Contenido es 0 para la radio y 1 para MP3 */
        WEB_OUTPUT_SEL, /**< Cambio de salida. Contenido es 0 para cascos y 1 para altavoz */
        WEB_LOW_POWER,  /**< Entrar en modo bajo consumo. Contenido ignorado */
        WEB_RADIO_FREQ, /**< Cambio de frecuencia de la radio. Contenido es la frecuencia en centenas de kHz */
        WEB_SEEK,       /**< Hacer seek con la radio. Contenido es 0 para down y 1 para up */
        WEB_VOL,        /**< Cambio de volumen. Contenido es el volumen */
        WEB_SONG,       /**< Cambio de cancion. Contenido es el numero de cancion */
        WEB_PLAY_PAUSE, /**< Alternar play y pause de la web */
        WEB_PREV_SONG,  /**< Anterior cancion */
        WEB_NEXT_SONG,  /**< Siguiente cancion*/
        WEB_BANDS,      /**< Cambio de filtro. Contenido es primer byte la banda [0,4] segundo la cantidad [-9, 9] */
        WEB_SAVE_SD,    /**< Guardar configuracion en la SD. Contenido ignorado */
        WEB_LOOP,       /**< Poner cancion en bucle. Contenido ignorado*/
    } web_msg_type_t;

    /**
    * @brief Estructura para los mensajes de entrada del LCD
    */
    typedef struct {
        web_msg_type_t type;    /**< Tipo del mensaje de */
        uint16_t       payload; /**< Contenido del mensaje. Depende del tipo */
    } web_msg_t;

    // ====================================== NFC ========================================
    /**
    * @brief Estructura para los mensajes de entrada del NFC
    */
    typedef struct {
        uint8_t  type;    /**< Tipo de mensaje. 0 para cancion y 1 para radio*/
        uint17_t content; /**< Numero de cancion o frecuencia en centenas*/
    } nfc_msg_t;

    // ====================================== RTC ========================================
    /**
    * @brief Estructura para los mensajes de entrada del RTC
    */
    typedef struct {
        uint8_t hour, minute, second, day, month, year;
    } rtc_msg_t;

    // ====================================== MSG ========================================
    /**
    * @brief Estructura para los mensajes de entrada
    */
    typedef struct {
        msg_ctrl_type_t type;    /**< Tipo de mensaje de entrada. 
                                    Dependiendo de este valor se debe interpretar el contenido */
        union {
            nfc_msg_t nfc_msg;   /**< Contenido de un mensaje de tipo MSG_NFC   */
            lcd_msg_t lcd_msg;   /**< Contenido de un mensaje de tipo MSG_LCD   */
            rtc_msg_t rtc_msg;   /**< Contenido de un mensaje de tipo MSG_RTC   */
            web_msg_t web_msg;   /**< Contenido de un mensaje de tipo MSG_WEB   */
            uint16_t  cons_msg;  /**< Contenido de un mensaje de tipo MSG_CONS  */
            uint32_t  radio_msg; /**< Contenido de un mensaje de tipo MSG_RADIO */
        };
    } msg_ctrl_t;

    #endif
    
\end{lstlisting}