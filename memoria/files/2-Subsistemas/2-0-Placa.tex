\subsection{La placa}
Para este proyecto se ha utilizado la placa \texttt{STM32F769NI-DISCO}, que es una placa de desarrollo comercializada por STM. Dispone de un procesador ARM Cortex M7, que puede funcionar a una frecuencia de hasta 216 MHz. 

Dispone de pines dispuestos de una forma adecuada para ser compatible con Arduino Uno.

Entre los periféricos presentes en la placa, los más importantes que utilizamos son:
\begin{itemize}
    \item \textbf{La pantalla:} Dispone de una pantalla LCD táctil de 4.3 pulgadas, accesible a través del periférico DSI, con una resolución de 800 x 480 píxeles, con una profundidad del  color de 16 bits, lo que resulta en un total de $2^{16}$ colores posibles de representar. En concreto, se asignan 5 bits para las componentes roja y azul, siendo los 6 restantes para la verde, que es la más importante.
    \item \textbf{La DMA2:} La pantalla restringe el uso de las DMA, siendo la 2 la única opción posible, puesto que es la única que tiene la opción de \textit{Memory-To-Memory}. Cualquier canal y flujo (\textit{Channel} y \textit{Stream}) son de libre elección.
    \item \textbf{Los ADCs:} La placa dispone tres ADCs, que funcionan a una frecuencia de 48 kHz, de los cuales nuestro proyecto utiliza dos: el ADC1 se utiliza para medir el consumo, y el ADC3 se utiliza como pin de entrada para el audio. 
    \item \textbf{El DAC:} La placa dispone de un DAC de 2 canales. En la documentación de la placa no se anuncia esta capacidad, únicamente se puede encontrar el la documentación del microcontrolador. El DAC también funciona a 48kHz, y se sincroniza con el ADC, como se explica en \autoref{para:temporizacion}.
    \item \textbf{El I2C:} Utilizamos un I2C para la comunicación con el módulo del NFC.
    \item \textbf{La SD:} Dispone de un puerto para introducir una tarjeta SD, a través del periférico SDMMC.
\end{itemize} 

\subsubsection{Algunas consideraciones}
En algunos proyectos, hemos intentado utilizar la sentencia \texttt{printf}, pero a priori parecía una tarea imposible. Revisando la documentación, constatamos que el pin \texttt{SW0} está desconectado en la configuración por defecto. Para que funcione dicha función, hay que soldar \texttt{R92}, que conecta \texttt{SW0} con el pin adecuado. Esta resistencia funciona como un jumper, ya que el valor anunciado por el fabricante es de 0 $\Omega$. Tras sujetar un cable de forma manual, conseguimos visualizar la salida de la \texttt{UART} desde el panel del debugger.